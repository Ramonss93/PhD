% !TEX root=ProposalJavier.tex 
% the previous is to reference main .bib
%% CHAPTER
\chapter{Background and Theory}
\label{ch:background}
% -----------------------------------------------------------------------------
\section{IOPs}

Absorption Coefficient.
Scattering Coefficient.

bb, Bb

VSF

Volumen Scattering Phase Function, or $\beta{hat}$ 
% -----------------------------------------------------------------------------
\section{AOPs}
Remote Sensing Reflectance
% -----------------------------------------------------------------------------
\section{Landsat-8}
Landsat-5, Landsat-7
MODIS, VIRSS
% -----------------------------------------------------------------------------
\section{Water Components}
Pure Water. Chl (pigments). CDOM (or DOM). SM or minerals.
% -----------------------------------------------------------------------------
\section{Retrieval Algorithms}
Invertion algorithm 
% -----------------------------------------------------------------------------
\section{Atmospheric Correction}
Atmospheric Correction used for MODIS, VIIRS...
ELM based develop. Band Ratio

Explain CDR reflectance product.
% -----------------------------------------------------------------------------
\section{MODTRAN}
% -----------------------------------------------------------------------------
\section{HydroLight}
HydroLight is a radiative transfer numerical model written in Fortran \cite{MobleHE}. It computes radiance distributions and derived quantities (e.g. irradiances, reflectances, K functions, etc.) for natural water bodies. It was developed by Dr. Curtis Mobley for over 20 years (since 1989) and is a commercial software product of Squoia Scientific, Inc.

% \begin{figure}[H]
% \begin{columns}[onlytextwidth] % contents are top vertically aligned
% 	\column{.35\textwidth}
%   		\includegraphics[height=3cm]{/Users/javier/Desktop/Javier/PHD_RIT/20123_Spring/Modeling/HydroLight/Beamer/absvsz.png}
%   	\column{.32\textwidth}
%   	\footnotesize
%   		\begin{equation}
%   			\cos\theta\frac{dL(z,\theta,\varphi,\lambda)}{dz}=\cdots \notag
%   		\end{equation}
% 	\column{.35\textwidth} 
% 		\includegraphics[height=3cm]{/Users/javier/Desktop/Javier/PHD_RIT/20123_Spring/Modeling/HydroLight/Beamer/RadSpec.png}
% \end{columns}
% \end{figure}
\todo{Try to fix fig.} 
\begin{figure}
% \resizebox{1.0\textwidth}{!}{%
	\centering
  \begin{tikzpicture}[node distance=0.75cm, auto]
          \tikzset{
                  basenode/.style={rectangle,rounded corners,draw=black,very thick, inner sep=1em, minimum size=3em, text centered,text width=2cm},
                  productnode/.style={ellipse,rounded corners,draw=black, very thick, text centered,text width=1.5cm},
                  myarrow/.style={->,>=stealth',thick, double = black},
                  mylabel/.style={text width=7em, text centered}
          }
          %\path[use as bounding box] (0,6) rectangle (4,2);
          \node[basenode] (IOPs) {Inherent Optical Properties};
          \node[basenode, below=of IOPs] (BC) {Boundary Conditions};
          \node[basenode, right=of IOPs] (RTE) {Radiative Transfer Equation};
          \node[basenode, right=of RTE] (rad) {Radiance Distribution};

          \draw[myarrow] (IOPs)--(RTE);
          \draw[myarrow] (BC)-|(RTE);
          \draw[myarrow] (RTE)--(rad);
  \end{tikzpicture}
% } %xobeziser
\caption{Hydrolight flow chart \label{fig:HLflowchart} } 
\end{figure}
\todo{talk about this figure}

The HydroLight physical model has the following characteristics:

\begin{itemize}
	\item It is time-independent.
	\item Horizontally homogeneous IOPs and boundary conditions $\Rightarrow$ one spatial dimension (depth): no restriction on depth dependence of IOPs.
	\item Wavelength between 300 and 1000 nm.
	\item Finite or infinitely deep (non-Lambertian) water-column bottom.
	\item Arbitrary sky radiance onto sea surface.
	\item Cox-Munk air-water surface (parameterizes gravity and capillary waves via the wind speed)
	\item Various bottom boundary options.
	\item Includes all orders of multiple scattering.
	\item It can optionally include Raman scatter by water.
	\item It can optionally include fluorescence by Chl and CDOM.
	\item It can optionally include horizontally homogeneous internal sources such as bioluminescing layers.
	\item Polarization not included.
\end{itemize}

\begin{figure}[H]
	\centering
	\includegraphics[height=6cm]{/Users/javier/Desktop/Javier/PHD_RIT/20123_Spring/Modeling/HydroLight/Beamer/RadianceDef.png}
\caption{Radiance (Figure from \cite{Mobley:2001}) \label{fig:radiance} } 
\end{figure}
\todo{can I use a fig. from other author?} 
where:\\
			\noindent $\Delta Q$: radian energy incident \\
			$\Delta t$: time interval \\
			$\Delta A$: surface area at location (x,y,z)\\
			$\Delta\Omega$: solid angle in direction ($\theta$,$\varphi$) \\
			$\Delta\lambda$: photons wavelength interval
% -----------------------------------------------------------------------------

\begin{equation}
	L(x,y,z,t,\theta,\varphi,\lambda)\equiv\frac{\Delta Q}{\Delta t\Delta A\Delta\Omega\Delta\lambda}~~\left[ Js^{-1}m^{-2}sr^{-1}nm^{-1} \right]
\end{equation}

\begin{equation}
	L(x,y,z,t,\theta,\varphi,\lambda)\equiv\frac{\partial^4 Q}{\partial t\partial A\partial\Omega\partial\lambda}~~\left[ Js^{-1}m^{-2}sr^{-1}nm^{-1} \right]
\end{equation}


{\bf Radiometric Quantities}
\textbf{Spectral downwelling scalar irradiance} at depth z:
\begin{equation}
	E_{od}(z,\lambda)=\int_{2\pi_d} L(z,\theta,\varphi,\lambda)d\Omega~~\left[Wm^{-2}nm^{-1} \right]
\end{equation}
\textbf{Spectral upwelling scalar irradiance} at depth z:
\begin{equation}
	E_{ou}(z,\lambda)=\int_{2\pi_u} L(z,\theta,\varphi,\lambda)d\Omega~~\left[Wm^{-2}nm^{-1} \right]
\end{equation}
\textbf{Spectral scalar irradiance} at depth z:
\begin{align}
	E_{o}(z,\lambda) &\equiv E_{od}(z,\lambda)+E_{ou}(z,\lambda)\\
					 &=\int_{4\pi} L(z,\theta,\varphi,\lambda)d\Omega
\end{align}

\textbf{Spectral downwelling plane irradiance} at depth z:
\begin{equation}
	E_{d}(z,\lambda)=\int_{2\pi_d} L(z,\theta,\varphi,\lambda)|cos\theta|d\Omega~~\left[Wm^{-2}nm^{-1} \right]
\end{equation}
Photosynthetic available radiation, \textbf{PAR}:
\begin{equation}
	PAR(z)\equiv \int_{350nm}^{700nm} \frac{\lambda}{hc}E_o(z,\lambda)d\lambda~~~\left[photons~s^{-1}m^{-2} \right]
\end{equation}

		\begin{figure}[H]
		\centering
		\includegraphics[height=3.5cm]{/Users/javier/Desktop/Javier/PHD_RIT/20123_Spring/Modeling/HydroLight/Beamer/IOPgeo.png}
		\caption{caption \label{label} } 
		\end{figure}

		\textbf{Absorption coefficient:}
		\begin{equation}
			a(\lambda)\equiv \lim_{\Delta r\to 0} \frac{1}{\Phi_i(\lambda)}\frac{\Phi_a(\lambda)}{\Delta r}~~\left[m^{-1} \right]
		\end{equation}
		\textbf{Scattering coefficient:}
		\begin{equation}
			b(\lambda)\equiv \lim_{\Delta r\to 0} \frac{1}{\Phi_i(\lambda)}\frac{\Phi_s(\lambda)}{\Delta r}~~\left[m^{-1} \right]
		\end{equation}
		\textbf{Attenuation coefficient:}
		\begin{equation}
			c(\lambda)=a(\lambda)+b(\lambda)~~\left[m^{-1} \right]
		\end{equation}

{\bf Inherent Optical Properties}
Contributions by the various components to the absorption coefficient
\begin{figure}[H]
\centering
  		\includegraphics[height=5cm]{/Users/javier/Desktop/Javier/PHD_RIT/20123_Spring/Modeling/HydroLight/Beamer/AbsCoeff.png}
\end{figure}


\begin{figure}[H]
\centering
  		\includegraphics[height=5cm]{/Users/javier/Desktop/Javier/PHD_RIT/20123_Spring/Modeling/HydroLight/Beamer/ScatCoeff.png}
\end{figure}


Volume scattering function, \textbf{VSF}:
\begin{equation}
	\beta(\psi,\lambda)\equiv \lim_{\Delta r\to 0} \lim_{\Delta \Omega\to 0}  \frac{\Phi_s(\psi,\lambda)}{\Phi_i(\lambda)\Delta r\Delta \Omega}~~\left[m^{-1}sr^{-1} \right]
\end{equation}
but $\Phi_s(\psi,\lambda)=I_s(\psi,\lambda)\Delta \Omega$ and $E_i(\lambda)=\Phi_i(\lambda)/\Delta A \Rightarrow$
\begin{equation}
	\beta(\psi,\lambda)= \lim_{\Delta V\to 0} \frac{I_s(\psi,\lambda)}{E_i(\lambda)\Delta V}~~\left[m^{-1}sr^{-1} \right]
\end{equation}
with $\Delta V=\Delta r\Delta A$

Scattering coefficient from VSF:
\begin{equation}
	b(\lambda)=\int_{4\pi} \beta(\psi,\lambda)d\Omega=2\pi\int_0^\pi \beta(\psi,\lambda)sin\psi d\psi
\end{equation}
And \textbf{backscatter coefficient}:
\begin{equation}
	b_b(\lambda)\equiv 2\pi\int_{\pi/2}^\pi \beta(\psi,\lambda)sin\psi d\psi
\end{equation}


\begin{figure}[H]
\centering
  		\includegraphics[height=4cm]{/Users/javier/Desktop/Javier/PHD_RIT/20123_Spring/Modeling/HydroLight/Beamer/VSFweb.png}
\end{figure}
Example VSFs:\\
\textcolor{blue}{blue curve} : clear, open ocean water ($\lambda = 514 nm$)\\
\textcolor{green}{green curve} : harbor ($\lambda = 514 nm$)\\
\textcolor{red}{red curve} : very productive coastal waters ($\lambda = 530 nm$)

\textbf{Backscatter fraction:}
\begin{equation}
	B_b=\frac{b_b}{b}
\end{equation}
\textbf{Single-scattering albedo:}
\begin{equation}
	\omega_o=\frac{b(\lambda)}{c(\lambda)}
\end{equation}
\textbf{Volumen scattering phase function:}
\begin{equation}
	\tilde{\beta}(\psi,\lambda)\equiv \frac{\beta(\psi,\lambda)}{b(\lambda)}~~\left[sr^{-1} \right]\Rightarrow \beta(\psi,\lambda)=b(\lambda)\tilde{\beta}(\psi,\lambda)
\end{equation}

{Apparent Optical Properties}
\textbf{AOPs:} depend both on the medium and on the directional structure of the ambient light field

\vspace{\baselineskip}

\textbf{Irradiance reflectance:}
\begin{equation}
	R(z,\lambda)\equiv \frac{E_u(z,\lambda)}{E_d(z,\lambda)}
\end{equation}
\textbf{Remote sensing reflectance:}
\begin{equation}
	R_{rs}(\theta,\varphi,\lambda)\equiv \frac{L_w(\theta,\varphi,\lambda)}{E_d(\lambda)}~~\left[sr^{-1} \right]
\end{equation}
where $L_w$ is the \textbf{water-leaving radiance}

Under typical oceanic conditions, irradiances and radiances decrease exponentially with depth, then
\begin{equation}
	E_d(z,\lambda)\equiv E_d(0,\lambda) exp\left[-\int_0^{z}K_d(z',\lambda)dz'\right]
\end{equation}
where $K_d(z,\lambda)$ is the \textbf{spectral diffuse attenuation coefficient} for spectral downwelling plane irradiance. 
\begin{align}
	K_d(z,\lambda) 	&=- \frac{d\ln E_d(z,\lambda)}{dz} \notag \\
					&=-\frac{1}{E_d(z,\lambda)}\frac{dE_d(z,\lambda)}{dz} ~~\left[m^{-1} \right]
\end{align}

\textbf{Note:} IOPs are additive whereas AOPs are not

{Optical Constituents of Water}
\begin{itemize}
	\item Water
	\item Dissolved Organic Compounds (CDOM)
	\item Organic Particles (Chlorophyll)
	\item Inorganic Particles (Suspended Matter)
\end{itemize}

\textbf{Total IOP:}

\begin{equation}
	a_{total}(z,\lambda) = \sum_{i=1}^{ncomp} a_i(z,\lambda)
\end{equation}
e.g.,
\begin{equation}
	a_{total}(z,\lambda) =  a_w(\lambda) +a_{CDOM}(z,\lambda)+ a_{Chl}(z,\lambda)+a_{SM}(z,\lambda) \notag
\end{equation}

{The Radiative Transfer Equation}
\textbf{RTE} expresses conservation of energy for a collimated beam of radiance traveling through an absorbing, scattering and emitting medium

\vspace{\baselineskip}

The \textbf{RTE}:
\begin{align}
	\cos\theta\frac{dL(z,\theta,\varphi,\lambda)}{dz}&=-c(z,\lambda)L(z,\theta,\varphi,\lambda)\cdots \notag \\
	&+\int_{4\pi} L(z,\theta',\varphi',\lambda)\beta(z;\theta',\varphi' \to \theta,\varphi;\lambda)d\Omega'\cdots \notag  \\
	&+S(z,\theta,\varphi,\lambda)~~\left[W~m^{-3}sr^{-1}nm^{-1} \right]
\end{align}

\begin{figure}[H]
\centering
  		\includegraphics[height=4cm]{/Users/javier/Desktop/Javier/PHD_RIT/20123_Spring/Modeling/HydroLight/Beamer/RTE1.png}
\end{figure}

\begin{equation}
	-c(z,\lambda)L(z,\theta,\varphi,\lambda)\cdots \notag 
\end{equation}

\begin{figure}[H]
\centering
  		\includegraphics[height=4cm]{/Users/javier/Desktop/Javier/PHD_RIT/20123_Spring/Modeling/HydroLight/Beamer/RTE2.png}
\end{figure}

\begin{equation}
+\int_{4\pi} L(z,\theta',\varphi',\lambda)\beta(z;\theta',\varphi' \to \theta,\varphi;\lambda)d\Omega'\cdots \notag
\end{equation}

{\centering
\begin{equation}
	S(z,\theta,\varphi,\lambda) \notag
\end{equation}}

\begin{itemize}

\item {$S$ account for bioluminescence or inelastic-scattering processes}

\vspace{\baselineskip}

\item For example, the model is first run at the wavelength of excitation, $\lambda_{ex}$ to compute the energy shifted by inelastic scattering from $\lambda_{ex}$ to another wavelength $\lambda$, and the model is run again at $\lambda$, with the radiance shifted from $\lambda_{ex}$ appearing as a source term $S$ at $\lambda$

\end{itemize}

The \textbf{RTE}:
\begin{align}
	\cos\theta\frac{dL(z,\theta,\varphi,\lambda)}{dz}&=-c(z,\lambda)L(z,\theta,\varphi,\lambda)\cdots \notag \\
	&+\int_{4\pi} L(z,\theta',\varphi',\lambda)\beta(z;\theta',\varphi' \to \theta,\varphi;\lambda)d\Omega'\cdots \notag  \\
	&+S(z,\theta,\varphi,\lambda)~~\left[W~m^{-3}sr^{-1}nm^{-1} \right]
\end{align}
% ----------------------------------------------------------------------------------
\section{Water Quality Parameters Retrieval}
\cite{Jensen}
\cite{Mustard2001}
\subsection{Governing Equation}

\hl{The comparison is performed in the reflectance domain.}

\hl{CHL, SM and CDOM explanation}

From the total energy coming from the sun, only approximately 1390 $\left[\frac{W}{m^2}\right]$ reaches the Earth's atmosphere \cite{Schott}. This integrated value is known as the \emph{exoatmospheric irradiance}, or $E_S'$, and represents the total energy per unit area just outside the Earth's atmosphere due to the solar energy. Recall that \emph{irradiance} is the rate at which the radiant flux ($\Phi$) is delivered to a surface ($A$), defined as

\begin{equation} \label{eq:irradiance}
E = \frac{d\Phi}{dA}   \indent   \indent  \left[\frac{W}{m^2}\right]  
\end{equation} 

So, the $E_S'$ is calculated assuming that the flux $\Phi$ comes from a point source at the center of the sun such that it would produce an existence at the sun's surface, producing a flux at the mean Earth-sun distance of 1390 $\left[\frac{W}{m^2}\right]$ . For the present work, it is more convenient to express the irradiance spectrally, or other words as a function of wavelength, so we can describe the energy at desired wavelength, or spectral band.

The light source, usually the sun, interact with the target and then reach the sensor. This interaction will help us to extract about the target, in this case the water body. That is why in order to understand how the water quality parameter are retrieved, first it is necessary to introduce the concept of sensor reaching radiance. The sensor reaching radiance is defined as the accumulation of photons at the front of a sensor that one wishes to collect in an effort to obtain information about the target \cite{GeraceThesis} . The total sensor-reaching radiance is the sum of the radiances due to the individual solar and thermal paths.  \cite{Schott} shows that in the VNIR region (approximately 0.3-2.5 [$\mu m$]), the solar energy is so many orders of the magnitude higher than the self-emitted energy, so the thermal paths are negligible for this study. Also, we consider that radiance from the background is negligible because the water bodies are typically several kilometers wide. Assuming that the target is approximately Lambertian (radiance is equal in all directions), the total sensor-reaching radiance, $L$, is defined as

\begin{equation} \label{eq:gov1}s
L(\lambda) = \frac{E'_S(\lambda)cos(\sigma')r(\lambda)\tau_1(\lambda)\tau_2(\lambda)}{\pi} +
                        \frac{E_{ds}(\lambda)r(\lambda)\tau_2(\lambda)}{\pi} + L_{us}(\lambda)
\end{equation} 
where:
\begin{tabbing}
\indent \indent \indent  $L(\lambda)$ \hspace{1mm}\=:  \indent \= total sensor-reaching radiance\\
\indent \indent \indent  $E'_S(\lambda)$\>: \>exoatmospheric spectral irradiance\\
\indent \indent \indent $\sigma'$\>:\>solar-zenith angle\\
\indent \indent \indent $r(\lambda)$\>:\>spectral reflectance target\\
\indent \indent \indent $\tau_1(\lambda)$\>:\>Sun-target path transmission of atmosphere\\
\indent \indent \indent $\tau_2(\lambda)$\>:\>target-sensor path transmission of atmosphere\\
\indent \indent \indent $E_{ds}(\lambda)$\>:\>solar downwelling irradiance\\
\indent \indent \indent $L_{u}(\lambda)$\>:\>solar upwelling irradiance\\
\end{tabbing}

Equation \eqref{eq:gov1} is the solar term of the "big equation" described by \cite{Schott}.

\subsection{Constituent Retrieval}

When imaging water body, a set of new paths need to be taken in account to determine the sensor-reaching radiance.
 \begin{figure}[H]
	\centering
    	\includegraphics[width=100mm]{/Users/javier/Desktop/Javier/MASTER_RIT/2011_THESIS/LaTeX/Thesis/Images/WaterColumn.eps}
 	\caption{Contributions to sensor-reaching radiance from the water column \label{fig:WaterColumn}}
  \end{figure}


Each sensor-reaching radiance curve is associated with a specific combination of water components (CHL, SM and CDOM). HidroLight provides us the \hl{remote sensing reflectance}, $R_{RS}$, which describes how much of the total incident downwelling irradiance is ultimately returned from a water column in a given direction, defined as

\begin{equation} \label{eq:Rrs}
R_{RS}(\theta,\phi,\lambda,z=a) = \frac{L(\theta,\phi,\lambda,z=a)}{E_d(\lambda,z=a)}   \indent   \indent  \left[\frac{1}{sr}\right]  
\end{equation} 
where:
\begin{tabbing}
\indent \indent \indent  $\theta$ \hspace{1.5mm}\=:  \indent \= sensor-zenith angle\\
\indent \indent \indent  $\phi$\>: \>sensor-azimuth angle\\
\indent \indent \indent $L$\>:\>water-leaving radiance\\
\indent \indent \indent $E_d$\>:\>total downwelling irradiance\\
\indent \indent \indent $\lambda$\>:\>wavelength dependent\\
\indent \indent \indent $a$\>:\>height just above the water's surface\\
\end{tabbing}



\section{Shallow Water Retrieval}
\section{LUT Method}
\subsection{Population}
\subsection{Optimization Algorithm  - lsqnonlin}

\section{State of the Research}

As shown in \cite{GeraceThesis}
and \cite{Mobley} and \cite{Lesser}

\section{Atmospheric Compensation}