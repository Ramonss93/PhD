% !TEX root=ProposalJavier.tex 
% the previous is to reference main .bib
\chapter{Objectives}
\label{ch:objectives}
\missingfigure{add picture of the process here} 
As discussed in Chapter \ref{ch:introduction},  the retrieval of water constituent concentration using multispectral satellite imagery in a effort to monitor fresh and coastal water (referred to as Case 2 waters) is a complex problem because there is not a direct relationship between pixel values and water constituent concentration. However, since this problem possesses different links that depend on each other, it can be addressed in smaller tasks to make it easier to solve. 

The purpose of this chapter is precisely to define these tasks as an outline that will help to make the problem manageable. This chapter is divided in four sections in an effort to describe each of these tasks. Section \ref{sec:problemstatement} details the problem being approached. In Section \ref{sec:objectives}, the problem is outlined in {\color{red} three} separate objectives as well as some future objectives. Section \ref{sec:tasks} describes the tasks needed to accomplish these objectives. Finally, this chapter closes with Section \ref{sec:contributiontofield} that delineates this work's original contribution to the field of remote sensing, imaging science and ocean optics. 
% -----------------------------------------------------------------
\section{Problem Statement}
\label{sec:problemstatement}
The hypothesis addressed in this thesis is the following: 

{ \bf ``The L8 sensor can be utilized to simultaneously quantify the concentration of water constituents (specifically chlorophyll, {\todo{check this word} \color{red} suspended solids}, and colored dissolved organic matter) in fresh and coastal waters.''} 

This leads to the goal of our work: to develop a process to retrieve water constituents from L8 imagery to evaluate this satellite performance. Specifically, the algorithm will be used over Case 2 water, which includes fresh and coastal water. The retrieval algorithm compares a water leaving reflectance with unknown concentrations to water leaving reflectance whose concentrations are known. Because the comparison is made in reflectance domain, the process first requires atmospherically correcting the L8 image and two approaches are investigated to do so. The first one is an ELM-based algorithm while the second one is a {\todo{check term!} band-ratio approach}.

% -----------------------------------------------------------------
\section{Statement of Objectives}
\label{sec:objectives}
The successful completion of this research effort will be marked by completion of the following primary requirements. Future objectives will be addressed if time permits.

\subsection{Primary Requirements:}
\begin{enumerate}
	\item Develop over-water atmospheric correction algorithms for L8 reflective imagery.
	\item Design a water constituent concentration retrieval algorithm that can be applied to a specific study area.
	\item {\todo{Check with Dr. Schott} Include a glint correction.} 
	\item Validate results by comparing with in-situ measurements and products from ocean color satellites.
	\item Demo this process to a second study site.
\end{enumerate}

\subsection{Future Objectives:}
\begin{enumerate}
	\item Integration with Hydrodynamics models \todo{continue numeration} .
	\item Make the processes and algorithms more user friendly 
\end{enumerate}
% -----------------------------------------------------------------
\section{Description of Tasks}
\label{sec:tasks}

\subsection{Primary Requirements:}
\begin{enumerate} 
	{\bf \item Develop over-water atmospheric correction algorithms for L8 reflective imagery.} 

The first objective in this research is to identify the best approach to atmospherically correct the type of dataset provided by the OLI sensor. Two methods will be investigated. The first method will be based on previous work made for simulated OLI data \cite{Gerace:2013,Gerace:2012,GeraceThesis,Pahlevan:2012} that consists in an ELM-based method that combines the Landsat reflectance product (Landsat Surface Reflectance CDR;\cite{LandsatCDR}) and a physics-based numerical model over water (HydroLight) to determine both the bright and dark pixel reflectance. The second method will be based on methods developed for ocean color satellite such as SeaWiFS, MODIS, and MERIS \cite{Gordon:1997}. These methods are based on the fact that the signal leaving the water does not contribute to the overall signal beyond the NIR part of the spectrum of light, so the signal reaching the sensor is caused only by atmospheric scattering.

	{\bf \item Design a water constituent concentration retrieval algorithm that can be applied to a specific study area.}

The retrieval algorithm is based on previous work done by \cite{Raqueno:2003} and \cite{GeraceThesis}. The water-leaving reflectance product obtained after atmospheric correction from the previous stage is used as input to the retrieval algorithm. Each pixel of reflectance product has an unknown concentration. A spectral matching technique is applied to predict this concentration by comparing the spectral shape of each pixel with the elements in a LUT. The spectral matching is made by a least square error minimization along with a trilinear interpolation. This utilizes a non-linear optimization code provided in the MATLAB software \todo{Name Matlab package} . The output of this process is a concentration mapping for each water constituent.

The LUT is generated using the "case 2" {\todo{check this word} \color{red}  algorithm} in HydroLight for different triplets of water constituent concentrations. In order to generate congruent result from HydroLight, the user needs to input IOPs characteristic of the water bodies to be studied. Consequently, IOPs measured spectrophotometrically in the lab from water samples are used as input to HydroLight along with backscattering measurements in the field. These measurements were collected when the L8 sensor passed over the area of study.

The area of study is the lake Ontario Rochester embayment that includes some nearby ponds (Long and Cranberry ponds), the Genessee River plume, the Irondequoit bay and part of Lake Ontario. This area was selected because it exhibits a wide range of variability in concentration of water constituents, so the retrieval algorithm can be tested with different scenarios.
 
	{\bf \item Validate results by comparing with in-situ measurements and products from ocean color satellites.}

The results from the retrieval process are validated by comparison with measurements taken from the water bodies being studied. Before this process, the measurements needed to be validated with measurements analyzed by a credible lab (Monroe County Environmental Laboratory). This comparison with this lab shows agreement between the measurements. 

For further validation, the results will be compared with products derived from ocean color satellites such as MODIS (e.g. MODIS Chl{\it a} product) if possible.

	{\bf \item {\todo{Check with Dr. Schott} Demo this process to a second study site}.}

After validation of the retrieval algorithm over the study area, the next step would be to make it applicable to a second study site. To do so, a more general LUT would be created with elements representative of the different water bodies present in both study sites.


\end{enumerate}


% \subsection{Primary Requirements}
\subsection{Future Objectives}
	\begin{enumerate}
			{\bf \item Integration with hydrodynamics models. \todo{continue numeration} } 

The next step would be to use the validated results from the retrieval process for training hydrodynamics models to predicts future behavior of the water bodies. This would be based on previous work made by \cite{Pahlevan:2012} and \cite{GeraceThesis}, who used concentration maps obtained from the retrieval process using satellite imagery to train hydrodynamic models. For example, the hydrodynamic model would allow us to monitor the dynamics of coastal/inland waters near river discharges. The maps of water constituent concentrations on the surface can be used to feed in the hydrodynamic models in order to calibrate them. 

			{\bf \item Make the processes and algorithms more user friendly} 

The retrieval process described here requires integration of different modules from different software and use of different programming languages. The next step would be to create a graphical user interface (GUI) in Python to make the process more user friendly, so that anyone with basic remote sensing knowledge could use the methods describe in this thesis. We suggest Python because it is a free platform.

	\end{enumerate}	

		

% -----------------------------------------------------------------
\subsection{Contribution to Field}
\label{subsec:contributiontofield}
This research will make several contributions to the field of remote sensing.

First, one important contribution is to demonstrate that Landsat satellites, which have been historically underestimated for the use of water quality measurements, could have a good performance in the estimation of water constituent concentrations.

Second, L8 was just launched in February 2013 and therefore there are few studies done about its performance so far, specially in its applications related to water assessments. Hence, this is the perfect time to investigate how its new upgrades will improve/impact our capability of retrieving water parameters. Therefore, this research will present one of the first results of L8 performance over water studies.  

Third, while there are other global water constituent concentrations products, Landsat provides a unique combination of temporal (16 days repeat cycle) and spatial resolution (30 m pixel size). Most of the retrieval algorithms available in the literature use ocean color satellites (e.g. SeaWiFS), which have spatial resolution of about 1 km to 250 m. Even though this resolution is suitable for large scale studies, they fail to cover small scale studies (less than 100 m). On the other hand, high spatial resolution sensors carried on aircraft (e.g. AVIRIS) or even satellite (e.g. WorldView-2) although they can be used for small scale studies, their imagery tends to be expensive or not frequently available. Here is where L8 has the potential of filling that gap because its spatial resolution (30 m) could allow study of medium size targets, a river plume, for instance, and it is free to the international scientific community.

This research also contributes to the field of remote sensing by developing a novel approach to correct the atmospheric effect in L8 images over Case 2 waters via two different approaches. In spite of the fact that the ELM method is widely used to correct satellite image, it needs measurements in the field that are not always available. We developed an algorithm that overcomes this issue by estimating these measurements. Additionally, the {\todo{correct name, is it band ratio?} \color{red} second method} tries to benefit from the concepts behind the methods largely developed and used to atmospherically correct ocean optics sensors.

Finally, a large dataset is made available for potential water quality studies through this research. L8 collects images all around the world where there is land including fresh and coastal waters. Such wide-reaching temporal and spatial coverage is not being broadly exploited for water quality studies.

The background material necessary to attain these goals is described in the following chapter.
