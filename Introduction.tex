% !TEX root=ProposalJavier.tex 
% the previous is to reference main .bib
%% CHAPTER
\chapter{Introduction}
\label{ch:introduction} 
\pagenumbering{arabic} 
Landsat satellites' main mission is to image the land areas of the earth and therefore there are not open ocean (case 1 water) images available. This is the reason why Landsat satellites have been underestimated by the ocean color community for the study of water bodies. This is mainly for its broad bands and low SNR when compared to ocean color satellites such as SeaWiFS and MODIS. However, L8 could fulfill a niche in the ocean community for coastal and in-land water  bodies studies where a 250 m pixel size satellite is not suitable. This is where L8 has a valuable potential for water quality studies in more optically complex water bodies (case 2 waters). Therefore, the overall objective of this research is to demonstrate that the new generation of Landsat satellites, Landsat-8 satellite, hereafter L8, is capable of accurately retrieving water constituents.

The Landsat project has been monitoring the earth for over four decades, being the longest uninterrupted data set available. L8 is the new satellite that continues with this objective. Carrying two instruments onboard, the Operational Land Imager (OLI) and the Thermal InfraRed Scanner (TIRS), L8 is the first of a new generation of Landsat satellite with state-of-the-art technology. With its 12-bit quantization and improved signal-to-noise ratio (SNR), OLI is a big improvement to the Landsat mission. In addition, OLI includes a new coastal band that increased the spectral resolution of the instrument. These improvements are the main drivers to hypothesize that the L8 satellite will definitely have a better performance in the water quality studies than its predecessors. 

The retrieval of water components is {\color{red} in general performed in reflectance domain}, so the very first step in this work is to perform a {\color{red} decent} atmospheric correction to the radiance image from L8. This is a complex task to perform over water because the signal leaving the water that reaches the sensor is too small when compared to the signal reaching the sensor produced by the atmospheric scattering. Most of the atmospheric correction algorithms applied to color ocean satellites are not suitable for highly turbid coastal water \cite{Patt2003}. In this work, two different approaches for atmospheric correction will be investigated. The first one is an ELM-based technique that uses a combination of a radiative transfer model over water and a Landsat reflectance product to determine the black and dark pixels in the image. The second one is an in-scene/band ratio approach, similar to the ones used for atmospherically correcting ocean color satellites (such as SeaWiFS). The water-leaving reflectance values obtained after atmospheric correction are validated by comparison with water surface reflectance measured in situ. 

After having corrected the image, the next step is to apply a retrieval algorithm that outputs water component retrieval maps of the main water components (Chlorophyll, sediment and CDOM). A spectral matching and look-up table (LUT) approach is utilized. It uses a least square error minimization algorithm to find the best match for a specific reflectance signal in a LUT of spectral water-leaving reflectance curves. The LUT is created using HydroLight 5, a radiative transfer model over water. Each curve in the LUT has a specific set of water component concentrations. This is performed in a pixel-by-pixel basis. The concentration values obtained from the retrieval algorithm are validated by comparison with concentration measured in lab from water bodies present in the L8 image.

In order to have outputs that are representative of the water bodies that are being studied, Inherent Optical Properties (IOPs) of those specific waters have to be input to the HydroLight model. To accomplish this, collections of water samples were conducted at the same time that the L8 satellite passed over the area of study (Rochester, NY). After collection, these water samples were analyzed in lab to obtain IOPs for the main water constituents. Furthermore, AOPs and backscattering measurements were also collected for further comparison and pursue closure between HydroLight AOPs results and in-situ AOPs measurements.

% \pagenumbering{arabic}