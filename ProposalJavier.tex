\documentclass{book}

\usepackage{graphicx}
\usepackage{float}
\usepackage{amssymb,amsmath}
\newcommand{\bm}[1]{\boldsymbol{#1}}

\usepackage{multirow}
\usepackage{fullpage}
\usepackage{appendix}
\usepackage{setspace}

\usepackage[bf, small, center]{caption}
\setlength{\belowcaptionskip}{10pt}
\usepackage{longtable}

\usepackage{geometry}
\geometry{
%  top=1.0in,  
%  inner=1.5in,
%  outer=1.5in,
  bottom=1.2in,
%  headheight=3ex, 
  headsep=3ex,         
}

\usepackage{fancyhdr}
\pagestyle{fancy}
\fancyhead[LO, RE]{\rightmark}
\fancyhead[LE, RO]{\thepage}
\fancyfoot[]{}
%\renewcommand{\headrulewidth}{0.3pt}

\usepackage{array}
\newcolumntype{L}[1]{>{\raggedright\let\newline\\\arraybackslash\hspace{0pt}}m{#1}}
\newcolumntype{C}[1]{>{\centering\let\newline\\\arraybackslash\hspace{0pt}}m{#1}}
\newcolumntype{R}[1]{>{\raggedleft\let\newline\\\arraybackslash\hspace{0pt}}m{#1}}



\usepackage[phd]{thesisfrontmatter}

% *** NOTE WHICH PROGRAM GENERATED IMAGES OR WHICH POWER POINT THEY ARE FROM
%compare methodology with script for final process
%how much detail in methodology?
%some sort of processing flowchart?
%how much do we care about processing?

\usepackage[parfill]{parskip}    % Activate to begin paragraphs with an empty line rather than an indent


\usepackage{mathtools}


\usepackage{color}
\usepackage{soul}



\DeclareGraphicsRule{.tif}{png}{.png}{`convert #1 `dirname #1`/`basename #1 .tif`.png}

\let\stdsection\chapter  
\renewcommand\chapter{\newpage\stdsection}  

%%%%%%%%%%% for Appendix
\makeatletter
\newcommand\appendix@section[1]{%
  \refstepcounter{section}%
  \orig@section*{Appendix \@Alph\c@section: #1}%
  \addcontentsline{toc}{section}{Appendix \@Alph\c@section: #1}%
}
\let\orig@section\section
\g@addto@macro\appendix{\let\section\appendix@section}
\makeatother

% Added 12-03-13 ----------------------------------------
\usepackage{imakeidx} % to create index
\indexsetup{othercode=\small}
\makeindex[program=makeindex,columns=2,intoc=true,options={-s MyIndex.ist}]
% -------------------------------------------------------
%%%%%%%%%%%%%%

\usepackage{hyperref}
\hypersetup{
    % bookmarks=true,         % show bookmarks bar?
    unicode=false,          % non-Latin characters in AcrobatÕs bookmarks
    pdftoolbar=true,        % show AcrobatÕs toolbar?
    pdfmenubar=true,        % show AcrobatÕs menu?
    pdffitwindow=false,     % window fit to page when opened
    pdfstartview={FitH},    % fits the width of the page to the window
    pdftitle={L8's potential for water constituents retrieval },    % title
    pdfauthor={Javier Concha},     % author
    pdfsubject={Subject},   % subject of the document
    pdfcreator={Creator},   % creator of the document
    pdfproducer={Producer}, % producer of the document
    pdfkeywords={keyword1} {key2} {key3}, % list of keywords
    pdfnewwindow=true,      % links in new window
    colorlinks=true,       % false: boxed links; true: colored links
    linkcolor=blue,          % color of internal links
    citecolor=cyan,        % color of links to bibliography
    filecolor=magenta,      % color of file links
    urlcolor=green           % color of external links
}

\usepackage[all]{hypcap} % to see figure with hyper ref

\setcounter{secnumdepth}{5}
\setcounter{tocdepth}{5}

% Added 11-12-13 ----------------------------------------

\usepackage{titlesec} % For the spacing after and before chapter title

\titleformat{\chapter}[display]
    {\normalfont\huge\bfseries}{\chaptertitlename\ \thechapter}{10pt}{\Huge}
\titlespacing*{\chapter}{0pt}{40pt}{20pt}

\titlespacing*{\section}{0ex}{2.5ex}{2ex}
\titlespacing*{\subsection}{0ex}{1.5ex}{1ex}

\usepackage{indentfirst}
\setlength{\parindent}{20pt}
\doublespacing

% Added 11-17-13 ----------------------------------------
% Select what to do with todonotes: 
% \usepackage[disable]{todonotes} % notes not showed
% \usepackage[draft]{todonotes}   % notes showed
% \setlength{\marginparwidth}{2cm}
\usepackage[textwidth=3.7cm]{todonotes}

\setlength{\marginparwidth}{3.7cm}

% Added 11-15-13 ----------------------------------------
% \includeonly{Introduction,Objectives,Background_and_Theory,Methodology,Results,Summary,Appendix} 
\includeonly{Methodology} 
\usepackage{tikz} % for flow charts
  \usetikzlibrary{shapes,arrows,positioning,shadows,calc}
\usepackage{colortbl}
\usepackage{caption}
\usepackage{subcaption}
\usepackage{multirow}

\listfiles

\setlength{\headheight}{15pt} % to avoid "Package Fancyhdr Warning: \headheight is too small (0.0pt): Make it at least 12.0pt."

% Added 02-18-14 ----------------------------------------
% \setlength{\abovecaptionskip}{-2ex}
\setlength{\belowcaptionskip}{-4ex} % space after caption
%******************************************************************************************************
\usepackage{epsfig}
\usepackage{epstopdf}
\begin{document}

\degreetitle{The Use of Landsat-8 for Monitoring of Fresh and Coastal Water}
\degreeauthor{Javier A. Concha}
\degreedate{\today{}}
\prevdegreeA{M.S. Rochester Institute of Technology, 2012}
\advisor{Dr. John R. Schott}
\memberA{Dr. Anthony Vodacek}
\memberB{Dr. Aaron Gerace}
\memberC{Dr. Emmett Ientilucci}
\memberD{Dr. Christy Tyler}

\makeproposaldeclaration
\makePHDproposalapproval
% \makecopyright


% \maketitle

\pagenumbering{roman}

% \chapter*{Abstract}
% \addcontentsline{toc}{chapter}{Abstract}

\begin{abstract}
\setlength{\parindent}{20pt}
The Landsat Data Continuity Mission (LDCM; a.k.a. Landsat-8), recently launched (February 2013), is the next generation of Landsat satellite and continues with more than 40 years of continuing imaging acquisition, playing a critical role in monitoring, understanding and managing natural resources such as water. Landsat-8, with its improved spectral bands and radiometric resolution, has the potential to dramatically improve our ability to simultaneously retrieve the three primary coloring agents, chlorophyll (Chl), colored dissolved organic material (CDOM) and suspended material (SM) from water bodies. This work presents the results obtained so far. To date, the method has been tested on Landsat-5 images because Landsat-8 images are still not available.

In the Case 2 water problem, the sensor-reaching signal due to water is very small when compared to the signal due to the atmospheric effects. Therefore, adequate atmospheric correction becomes an important first step to accurately retrieving water parameters. As a first approach, a model based empirical line method (ELM) atmospheric correction method converts sensor-reaching radiance to water leaving reflectance. This model employs pseudo invariant feature (PIF) pixels extraction from Landsat images along with an in water radiative transfer model (HydroLight) to obtain the field spectra. Further atmospheric compensation technique will be investigated.

A look-up-table (LUT) methodology is implemented to retrieve the water parameters. The LUT is created using HydroLight.

Collections of water samples when the satellite passes over the Rochester area are planned for summer 2013. Concentration and inherent optical properties (IOPs) measurement will help to validate the methods.

\end{abstract}

\chapter*{Acknowledgements}
%*****************************************************************************************************
\begin{acknowledgements}
\setlength{\parindent}{20pt}
To my parents, Osvaldo Concha and Nelly Sepulveda, sister Loretto Concha and brother Alvaro Concha. USGS. Monroe County Environment: Scott, Providencia and Gary. Nina and Rolo. Dr. Christy Taylor. Paul. Aaron Gerace. Alan Wiedmman. Ocean Color community: Emmanuel Boss and attendants of the 2013 Ocean color and instrumentation summer course at the Darling Marine Center. Fulbright commission. My advisor Dr. John Schott. Amanda and Cindy. Classmates: Aly, Bikash, Viraj, Madurima, Peter Sun. Professor Dr. John Kerekes. Collection help: Harold Valdivia. My dear friends: Juan Saldana, Kader, Patrick, Sarada, Susan, Melisa, Sasha, Jeremy. The people at Crossfit Rochester: Joe, Andrew. The Latin Rhythm Dance Club at RIT. Aixa de Jesus. My adopted Chilean family and dancers in Rochester: Marcia, Rosa and Rosita, Raul, Willy and Marisol, Harold and Nancy. My Chilean friends in U. of R. Felipe, Maria Eugenia, Emilio, Brenda. And my lovely girlfriend Kimberley Thoms.
\end{acknowledgements}
%*****************************************************************************************************

\tableofcontents

\listoffigures
\addcontentsline{toc}{chapter}{List of Figures}

\listoftables
\addcontentsline{toc}{chapter}{List of Tables}

% % !TEX root=ProposalJavier.tex 
% the previous is to reference main .bib
%% CHAPTER
\chapter{Introduction}
\label{ch:introduction} 
\pagenumbering{arabic} 
Landsat satellites' main mission is to image the land areas of the earth and therefore there are typically no open ocean (case 1 water) images available. This is the reason why Landsat satellites have been underestimated by the ocean color community for the study of water bodies. In addition, the Landsat instruments have generally had broad bands and low SNR when compared to ocean color satellites such as SeaWiFS and MODIS. However, Lansat-8 (L8) could fulfill a niche in the ocean community for coastal and in-land water  body studies where a 250 m pixel size satellite is not suitable. This is where L8 has a valuable potential for water quality studies in more optically complex water bodies (case 2 waters). Therefore, the overall objective of this research is to demonstrate that the new generation of Landsat satellites are capable of accurately retrieving water constituents.

The Landsat project has been monitoring the earth for over four decades, being the longest uninterrupted data set available. L8 is the new satellite that continues with this objective. Carrying two instruments onboard, the Operational Land Imager (OLI) and the Thermal InfraRed Scanner (TIRS), L8 is the first of a new generation of Landsat satellite with state-of-the-art technology. With its 12-bit quantization and improved signal-to-noise ratio (SNR), OLI is a big improvement to the Landsat mission. In addition, OLI includes a new coastal band that increased the spectral resolution of the instrument. These improvements are the main drivers to hypothesize that the L8 satellite will definitely have a better performance in water quality studies than its predecessors. 

The retrieval of water components is {\color{red} in general performed in the reflectance domain}, so the very first step in this work is to perform a {\color{red} decent} atmospheric correction to the radiance image from L8. This is a complex task to perform over water because the signal leaving the water that reaches the sensor is very small when compared to the signal reaching the sensor produced by the atmospheric scattering. Most of the atmospheric correction algorithms applied to color ocean satellites are not suitable for highly turbid coastal water \cite{Patt2003}. In this work, two different approaches for atmospheric correction will be investigated. The first one is an empirical line method-based (ELM-based) technique that uses a combination of a radiative transfer model over water and a Landsat reflectance product to determine the bright and dark pixels in the image. The second one is an {\todo{which one?} in-scene/band} ratio approach, similar to the ones used for atmospherically correcting ocean color satellites (such as SeaWiFS). The water-leaving reflectance values obtained after atmospheric correction are validated by comparison with water surface reflectance measured in situ. 

After having corrected the image, the next step is to apply a retrieval algorithm that outputs water component retrieval maps of the main water components (chlorophyll, {\todo{or SM?} sediment} and colored dissolved organic matter (CDOM)). A spectral matching and look-up table (LUT) approach is utilized. It uses a least square error minimization algorithm to find the best match for a specific reflectance signal in a LUT of spectral water-leaving reflectance curves. The LUT is created using HydroLight 5, an in-water radiative transfer model. Each curve in the LUT has a specific set of water component concentrations. This is performed on a pixel-by-pixel basis. The concentration values obtained from the retrieval algorithm are validated by comparison with concentration measured in the lab from water bodies present in the L8 image.

In order to have outputs that are representative of the water bodies that are being studied, Inherent Optical Properties (IOPs) of those specific waters have to be input to the HydroLight model. To accomplish this, collections of water samples were conducted at the same time that the L8 satellite passed over the area of study (Rochester, NY). After collection, these water samples were analyzed in the lab to obtain IOPs for the main water constituents. Furthermore, apparent optical properties (AOPs) and backscattering measurements were also collected for further comparison and to pursue closure between HydroLight AOPs results and in-situ AOPs measurements.

% \pagenumbering{arabic}
% !TEX root=ProposalJavier.tex 
% the previous is to reference main .bib
%% CHAPTER
\chapter{Introduction}
\label{ch:introduction} 
\pagenumbering{arabic} 
Landsat satellites' main mission is to image the land areas of the earth and therefore there are typically no open ocean (case 1 water) images available. This is the reason why Landsat satellites have been underestimated by the ocean color community for the study of water bodies. In addition, the Landsat instruments have generally had broad bands and low SNR when compared to ocean color satellites such as SeaWiFS and MODIS. However, Lansat-8 (L8) could fulfill a niche in the ocean community for coastal and in-land water  body studies where a 250 m pixel size satellite is not suitable. This is where L8 has a valuable potential for water quality studies in more optically complex water bodies (case 2 waters). Therefore, the overall objective of this research is to demonstrate that the new generation of Landsat satellites are capable of accurately retrieving water constituents.

The Landsat project has been monitoring the earth for over four decades, being the longest uninterrupted data set available. L8 is the new satellite that continues with this objective. Carrying two instruments onboard, the Operational Land Imager (OLI) and the Thermal InfraRed Scanner (TIRS), L8 is the first of a new generation of Landsat satellite with state-of-the-art technology. With its 12-bit quantization and improved signal-to-noise ratio (SNR), OLI is a big improvement to the Landsat mission. In addition, OLI includes a new coastal band that increased the spectral resolution of the instrument. These improvements are the main drivers to hypothesize that the L8 satellite will definitely have a better performance in water quality studies than its predecessors. 

The retrieval of water components is {\color{red} in general performed in the reflectance domain}, so the very first step in this work is to perform a {\color{red} decent} atmospheric correction to the radiance image from L8. This is a complex task to perform over water because the signal leaving the water that reaches the sensor is very small when compared to the signal reaching the sensor produced by the atmospheric scattering. Most of the atmospheric correction algorithms applied to color ocean satellites are not suitable for highly turbid coastal water \cite{Patt2003}. In this work, two different approaches for atmospheric correction will be investigated. The first one is an empirical line method-based (ELM-based) technique that uses a combination of a radiative transfer model over water and a Landsat reflectance product to determine the bright and dark pixels in the image. The second one is an {\todo{which one?} in-scene/band} ratio approach, similar to the ones used for atmospherically correcting ocean color satellites (such as SeaWiFS). The water-leaving reflectance values obtained after atmospheric correction are validated by comparison with water surface reflectance measured in situ. 

After having corrected the image, the next step is to apply a retrieval algorithm that outputs water component retrieval maps of the main water components (chlorophyll, {\todo{or SM?} sediment} and colored dissolved organic matter (CDOM)). A spectral matching and look-up table (LUT) approach is utilized. It uses a least square error minimization algorithm to find the best match for a specific reflectance signal in a LUT of spectral water-leaving reflectance curves. The LUT is created using HydroLight 5, an in-water radiative transfer model. Each curve in the LUT has a specific set of water component concentrations. This is performed on a pixel-by-pixel basis. The concentration values obtained from the retrieval algorithm are validated by comparison with concentration measured in the lab from water bodies present in the L8 image.

In order to have outputs that are representative of the water bodies that are being studied, Inherent Optical Properties (IOPs) of those specific waters have to be input to the HydroLight model. To accomplish this, collections of water samples were conducted at the same time that the L8 satellite passed over the area of study (Rochester, NY). After collection, these water samples were analyzed in the lab to obtain IOPs for the main water constituents. Furthermore, apparent optical properties (AOPs) and backscattering measurements were also collected for further comparison and to pursue closure between HydroLight AOPs results and in-situ AOPs measurements.

% \pagenumbering{arabic}
\pagenumbering{arabic} 
% !TEX root=ProposalJavier.tex 
% the previous is to reference main .bib
\chapter{Objectives}
\label{ch:objectives}

As discussed in Chapter \ref{ch:introduction},  the retrieval of water constituent concentration using multispectral satellite imagery in a effort to monitor fresh and coastal water (referred to as Case 2 waters) is a complex problem because there is not a direct relationship between pixel values and water constituent concentration. However, since this problem possesses different links that depend on each other, it can be addressed in smaller tasks to make it easier to solve. 

The purpose of this chapter is precisely to define these tasks as an outline that will help to make the problem manageable. This chapter is divided in four sections in an effort to describe each of these tasks. Section \ref{sec:problemstatement} details the problem being approached. In Section \ref{sec:objectives}, the problem is outlined in {\color{red} three} separate objectives as well as some future objectives. Section \ref{sec:tasks} describes the tasks needed to accomplish these objectives. Finally, this chapter closes with Section \ref{sec:contributiontofield} that delineates this work's original contribution to the field of remote sensing, imaging science and ocean optics. 
% -----------------------------------------------------------------
\section{Problem Statement}
\label{sec:problemstatement}
The hypothesis addressed in this thesis is the following: 

{ \bf ``The L8 sensor can be utilized to simultaneously quantify the concentration of the water color producing agents (CPAs) (specifically chlorophyll-{\it a}, {\todo{or minerals} sediment}, and colored dissolved organic matter) in fresh and coastal waters.''} 

This leads to the goal of our work: to develop a process to retrieve water constituents (CPAs) from L8 imagery to evaluate this satellite performance. Specifically, the algorithm will be used over Case 2 water, which includes fresh and coastal water. The retrieval algorithm compares a water leaving reflectance with unknown concentrations to water leaving reflectance whose concentrations are known. Because the comparison is made in reflectance domain, the process first requires atmospherically correcting the L8 image and two approaches are investigated to do so. The first one is an ELM-based algorithm while the second one is the standard SeaWiFS/MODIS algorithm using the shortwave infrared (SWIR) bands.

% -----------------------------------------------------------------
\section{Statement of Objectives}
\label{sec:objectives}
The successful completion of this research effort will be marked by completion of the following primary requirements. Future objectives will be addressed if time permits.

\subsection{Primary Requirements:}
\begin{enumerate}
	\item Develop over-water atmospheric correction algorithms for L8 reflective imagery.
	\item Design a water constituent concentration retrieval algorithm that can be applied to a specific study area.
	\item {\todo{Check with Dr. Schott} Include a glint correction.} 
	\item Validate results by comparing with in-situ measurements and products from ocean color satellites.
	\item Demo this process to a second study site.
\end{enumerate}

\subsection{Future Objectives:}
\begin{enumerate}
	\item Integration with Hydrodynamics models \todo{continue numeration} .
	\item Make the processes and algorithms more user friendly 
\end{enumerate}
% -----------------------------------------------------------------
\section{Description of Tasks}
\label{sec:tasks}

\subsection{Primary Requirements:}
\begin{enumerate} 
	{\bf \item Develop over-water atmospheric correction algorithms for L8 reflective imagery.} 

The first objective in this research is to identify the best approach to atmospherically correct the type of dataset provided by the OLI sensor. Two methods will be investigated. The first method will be based on previous work made for simulated OLI data \cite{Gerace:2013,Gerace:2012,GeraceThesis,Pahlevan:2012} that consists in an ELM-based method that combines the Landsat reflectance product (Landsat Surface Reflectance CDR;\cite{LandsatCDR}) and a physics-based numerical model over water (HydroLight) to determine both the bright and dark pixel reflectance. The second method will be based on methods developed for ocean color satellite such as SeaWiFS, MODIS, and MERIS \cite{Gordon:1997}. These methods are based on the fact that the signal leaving the water does not contribute to the overall signal beyond the NIR part of the spectrum of light (black pixel assumption), so the signal reaching the sensor is caused only by atmospheric scattering. This concept can be expanded to the SWIR bands when the black pixel assumption is not valid in the NIR bands, which is the case for Case 2 and high productive Case 1 waters (\cite{Wang:2007}).

	{\bf \item Design a water constituent concentration retrieval algorithm that can be applied to a specific study area.}

The retrieval algorithm is based on previous work done by \cite{Raqueno:2003} and \cite{GeraceThesis}. The water-leaving reflectance product obtained after atmospheric correction from the previous stage is used as input to the retrieval algorithm. Each pixel of reflectance product has an unknown concentration. A spectral matching technique is applied to predict this concentration by comparing the spectral shape of each pixel with the elements in a LUT. The spectral matching is made by a least square error minimization along with a trilinear interpolation. This utilizes a non-linear optimization code provided in Optimization Toolbox of the MATLAB software \todo{Name Matlab package}. The output of this process is a concentration mapping for each water constituent.

The LUT is generated using the ``Case 2'' \todo{check this word} algorithm in HydroLight for different triplets of water constituent concentrations. In order to generate congruent result from HydroLight, the user needs to input IOPs characteristic of the water bodies to be studied. Consequently, IOPs measured spectrophotometrically in the lab from water samples from the field are used as input to HydroLight along with backscattering measurements in the field. These measurements were collected when the L8 sensor passed over the area of study.

The area of study is the Lake Ontario Rochester Embayment that includes some nearby ponds (Long and Cranberry ponds), the Genessee River plume, the Irondequoit bay and part of Lake Ontario. This area was selected because it exhibits a wide range of variability in concentration of water constituents, so the retrieval algorithm can be tested with different scenarios.
 
	{\bf \item Validate results by comparing with in-situ measurements and products from ocean color satellites.}

The results from the retrieval process are validated by comparison with measurements taken from the water bodies being studied. Before this process, the measurements needed to be validated with measurements analyzed by a credible lab (Monroe County Environmental Laboratory). This comparison with this lab shows agreement between the measurements. 

For further validation, the results will be compared with products derived from ocean color satellites such as MODIS (e.g. MODIS Chl{\it a} product) if possible.

	{\bf \item {\todo{Check with Dr. Schott} Demo this process to a second study site}.}

After validation of the retrieval algorithm over the study area, the next step would be to make it applicable to a second study site. To do so, a more general LUT would be created with elements representative of the different water bodies present in both study sites.


\end{enumerate}


% \subsection{Primary Requirements}
\subsection{Future Objectives}
	\begin{enumerate}
			{\bf \item Integration with hydrodynamics models. \todo{continue numeration} } 

The next step would be to use the validated results from the retrieval process for training hydrodynamics models to predicts future behavior of the water bodies. This would be based on previous work made by \cite{Pahlevan:2012} and \cite{GeraceThesis}, who used concentration maps obtained from the retrieval process using satellite imagery to train hydrodynamic models. For example, the hydrodynamic model would allow us to monitor the dynamics of coastal/inland waters near river discharges. The maps of water constituent concentrations on the surface can be used to feed in the hydrodynamic models in order to calibrate them. 

			{\bf \item Make the processes and algorithms more user friendly} 

The retrieval process described here requires integration of different modules from different software and use of different programming languages. The next step would be to create a graphical user interface (GUI) in Python to make the process more user friendly, so that anyone with basic remote sensing knowledge could use the methods describe in this thesis. We suggest Python because it is a free platform.

	\end{enumerate}	

		

% -----------------------------------------------------------------
\section{Contribution to Field}
\label{sec:contributiontofield}
This research will make several contributions to the field of remote sensing.

First, one important contribution is to demonstrate that Landsat satellites, which have been historically underestimated for the use of water quality measurements, could have a good performance in the estimation of water constituent concentrations.

Second, L8 was just launched in February 2013 and therefore there are few studies done about its performance so far, specially in its applications related to water assessments. Hence, this is the perfect time to investigate how its new upgrades will improve/impact our capability of retrieving water parameters. Therefore, this research will present one of the first results of L8 performance over water studies.  

Third, while there are other global water constituent concentrations products, Landsat provides a unique combination of temporal (16 days repeat cycle) and spatial resolution (30 m pixel size). Most of the retrieval algorithms available in the literature use ocean color satellites (e.g. SeaWiFS), which have spatial resolution of about $1 km$ to $250 m$, but with products with \todo{check resolution}$4km$ spatial resolution (e.g. MODIS Chlorophyll-{\it a} product). Even though this resolution is suitable for large scale studies, they fail to cover small scale studies (less than 100 m). On the other hand, high spatial resolution sensors carried on aircraft (e.g. AVIRIS) or even satellite (e.g. WorldView-2) although they can be used for small scale studies, their imagery tends to be expensive or not frequently available. Here is where L8 has the potential of filling that gap because its spatial resolution (30 m) could allow study of medium size targets, a river plume, for instance, and it is free to the international scientific community.

This research also contributes to the field of remote sensing by developing a novel approach to correct the atmospheric effect in L8 images over Case 2 waters via two different approaches. In spite of the fact that the ELM method is widely used to correct satellite image, it needs measurements in the field that are not always available. We developed an algorithm that overcomes this issue by estimating these measurements. Additionally, the {\todo{correct name, is it band ratio?} \color{red} second method} tries to benefit from the concepts behind the methods largely developed and used to atmospherically correct ocean optics sensors.

Finally, a large dataset is made available for potential water quality studies through this research. L8 collects images all around the world where there is land including fresh and coastal waters. Such wide-reaching temporal and spatial coverage is not being broadly exploited for water quality studies.

The background material necessary to attain these goals is described in the following chapter.

% !TEX root=ProposalJavier.tex 
% the previous is to reference main .bib
%% CHAPTER
\chapter{Background and Theory}
\label{ch:background}
% -----------------------------------------------------------------------------
\section{IOPs}

Absorption Coefficient.
Scattering Coefficient.

bb, Bb

VSF

Volumen Scattering Phase Function, or $\beta{hat}$ 
% -----------------------------------------------------------------------------
\section{AOPs}
Remote Sensing Reflectance
% -----------------------------------------------------------------------------
\section{Landsat-8}
Landsat-5, Landsat-7
MODIS, VIRSS
% -----------------------------------------------------------------------------
\section{Water Components}
Pure Water. Chl (pigments). CDOM (or DOM). SM or minerals.
% -----------------------------------------------------------------------------
\section{Retrieval Algorithms}
Invertion algorithm 
% -----------------------------------------------------------------------------
\section{Atmospheric Correction}
Atmospheric Correction used for MODIS, VIIRS...
ELM based develop. Band Ratio

Explain CDR reflectance product.
% -----------------------------------------------------------------------------
\section{MODTRAN}
% -----------------------------------------------------------------------------
\section{HydroLight}
HydroLight is a radiative transfer numerical model written in Fortran \cite{MobleHE}. It computes radiance distributions and derived quantities (e.g. irradiances, reflectances, K functions, etc.) for natural water bodies. It was developed by Dr. Curtis Mobley for over 20 years (since 1989) and is a commercial software product of Squoia Scientific, Inc.

% \begin{figure}[H]
% \begin{columns}[onlytextwidth] % contents are top vertically aligned
% 	\column{.35\textwidth}
%   		\includegraphics[height=3cm]{/Users/javier/Desktop/Javier/PHD_RIT/20123_Spring/Modeling/HydroLight/Beamer/absvsz.png}
%   	\column{.32\textwidth}
%   	\footnotesize
%   		\begin{equation}
%   			\cos\theta\frac{dL(z,\theta,\varphi,\lambda)}{dz}=\cdots \notag
%   		\end{equation}
% 	\column{.35\textwidth} 
% 		\includegraphics[height=3cm]{/Users/javier/Desktop/Javier/PHD_RIT/20123_Spring/Modeling/HydroLight/Beamer/RadSpec.png}
% \end{columns}
% \end{figure}
\todo{Try to fix fig.} 
\begin{figure}
% \resizebox{1.0\textwidth}{!}{%
	\centering
  \begin{tikzpicture}[node distance=0.75cm, auto]
          \tikzset{
                  basenode/.style={rectangle,rounded corners,draw=black,very thick, inner sep=1em, minimum size=3em, text centered,text width=2cm},
                  productnode/.style={ellipse,rounded corners,draw=black, very thick, text centered,text width=1.5cm},
                  myarrow/.style={->,>=stealth',thick, double = black},
                  mylabel/.style={text width=7em, text centered}
          }
          %\path[use as bounding box] (0,6) rectangle (4,2);
          \node[basenode] (IOPs) {Inherent Optical Properties};
          \node[basenode, below=of IOPs] (BC) {Boundary Conditions};
          \node[basenode, right=of IOPs] (RTE) {Radiative Transfer Equation};
          \node[basenode, right=of RTE] (rad) {Radiance Distribution};

          \draw[myarrow] (IOPs)--(RTE);
          \draw[myarrow] (BC)-|(RTE);
          \draw[myarrow] (RTE)--(rad);
  \end{tikzpicture}
% } %xobeziser
\caption{Hydrolight flow chart \label{fig:HLflowchart} } 
\end{figure}
\todo{talk about this figure}

The HydroLight physical model has the following characteristics:

\begin{itemize}
	\item It is time-independent.
	\item Horizontally homogeneous IOPs and boundary conditions $\Rightarrow$ one spatial dimension (depth): no restriction on depth dependence of IOPs.
	\item Wavelength between 300 and 1000 nm.
	\item Finite or infinitely deep (non-Lambertian) water-column bottom.
	\item Arbitrary sky radiance onto sea surface.
	\item Cox-Munk air-water surface (parameterizes gravity and capillary waves via the wind speed)
	\item Various bottom boundary options.
	\item Includes all orders of multiple scattering.
	\item It can optionally include Raman scatter by water.
	\item It can optionally include fluorescence by Chl and CDOM.
	\item It can optionally include horizontally homogeneous internal sources such as bioluminescing layers.
	\item Polarization not included.
\end{itemize}

\begin{figure}[H]
	\centering
	\includegraphics[height=6cm]{/Users/javier/Desktop/Javier/PHD_RIT/20123_Spring/Modeling/HydroLight/Beamer/RadianceDef.png}
\caption{Radiance (Figure from \cite{Mobley:2001}) \label{fig:radiance} } 
\end{figure}
\todo{can I use a fig. from other author?} 
where:\\
			\noindent $\Delta Q$: radian energy incident \\
			$\Delta t$: time interval \\
			$\Delta A$: surface area at location (x,y,z)\\
			$\Delta\Omega$: solid angle in direction ($\theta$,$\varphi$) \\
			$\Delta\lambda$: photons wavelength interval
% -----------------------------------------------------------------------------

\begin{equation}
	\small L(x,y,z,t,\theta,\varphi,\lambda)\equiv\frac{\Delta Q}{\Delta t\Delta A\Delta\Omega\Delta\lambda}~~\left[ Js^{-1}m^{-2}sr^{-1}nm^{-1} \right]
\end{equation}

\begin{equation}
	\small L(x,y,z,t,\theta,\varphi,\lambda)\equiv\frac{\partial^4 Q}{\partial t\partial A\partial\Omega\partial\lambda}~~\left[ Js^{-1}m^{-2}sr^{-1}nm^{-1} \right]
\end{equation}


{\bf Radiometric Quantities}
\textbf{Spectral downwelling scalar irradiance} at depth z:
\begin{equation}
	E_{od}(z,\lambda)=\int_{2\pi_d} L(z,\theta,\varphi,\lambda)d\Omega~~\left[Wm^{-2}nm^{-1} \right]
\end{equation}
\textbf{Spectral upwelling scalar irradiance} at depth z:
\begin{equation}
	E_{ou}(z,\lambda)=\int_{2\pi_u} L(z,\theta,\varphi,\lambda)d\Omega~~\left[Wm^{-2}nm^{-1} \right]
\end{equation}
\textbf{Spectral scalar irradiance} at depth z:
\begin{align}
	E_{o}(z,\lambda) &\equiv E_{od}(z,\lambda)+E_{ou}(z,\lambda)\\
					 &=\int_{4\pi} L(z,\theta,\varphi,\lambda)d\Omega
\end{align}

\textbf{Spectral downwelling plane irradiance} at depth z:
\begin{equation}
	E_{d}(z,\lambda)=\int_{2\pi_d} L(z,\theta,\varphi,\lambda)|cos\theta|d\Omega~~\left[Wm^{-2}nm^{-1} \right]
\end{equation}
Photosynthetic available radiation, \textbf{PAR}:
\begin{equation}
	PAR(z)\equiv \int_{350nm}^{700nm} \frac{\lambda}{hc}E_o(z,\lambda)d\lambda~~~\left[photons~s^{-1}m^{-2} \right]
\end{equation}

		\begin{figure}[H]
		\centering
		\includegraphics[height=3.5cm]{/Users/javier/Desktop/Javier/PHD_RIT/20123_Spring/Modeling/HydroLight/Beamer/IOPgeo.png}
		\caption{caption \label{label} } 
		\end{figure}

		\textbf{Absorption coefficient:}
		\begin{equation}
			a(\lambda)\equiv \lim_{\Delta r\to 0} \frac{1}{\Phi_i(\lambda)}\frac{\Phi_a(\lambda)}{\Delta r}~~\left[m^{-1} \right]
		\end{equation}
		\textbf{Scattering coefficient:}
		\begin{equation}
			b(\lambda)\equiv \lim_{\Delta r\to 0} \frac{1}{\Phi_i(\lambda)}\frac{\Phi_s(\lambda)}{\Delta r}~~\left[m^{-1} \right]
		\end{equation}
		\textbf{Attenuation coefficient:}
		\begin{equation}
			c(\lambda)=a(\lambda)+b(\lambda)~~\left[m^{-1} \right]
		\end{equation}

{\bf Inherent Optical Properties}
Contributions by the various components to the absorption coefficient
\begin{figure}[H]
\centering
  		\includegraphics[height=5cm]{/Users/javier/Desktop/Javier/PHD_RIT/20123_Spring/Modeling/HydroLight/Beamer/AbsCoeff.png}
\end{figure}


\begin{figure}[H]
\centering
  		\includegraphics[height=5cm]{/Users/javier/Desktop/Javier/PHD_RIT/20123_Spring/Modeling/HydroLight/Beamer/ScatCoeff.png}
\end{figure}


Volume scattering function, \textbf{VSF}:
\begin{equation}
	\beta(\psi,\lambda)\equiv \lim_{\Delta r\to 0} \lim_{\Delta \Omega\to 0}  \frac{\Phi_s(\psi,\lambda)}{\Phi_i(\lambda)\Delta r\Delta \Omega}~~\left[m^{-1}sr^{-1} \right]
\end{equation}
but $\Phi_s(\psi,\lambda)=I_s(\psi,\lambda)\Delta \Omega$ and $E_i(\lambda)=\Phi_i(\lambda)/\Delta A \Rightarrow$
\begin{equation}
	\beta(\psi,\lambda)= \lim_{\Delta V\to 0} \frac{I_s(\psi,\lambda)}{E_i(\lambda)\Delta V}~~\left[m^{-1}sr^{-1} \right]
\end{equation}
with $\Delta V=\Delta r\Delta A$

Scattering coefficient from VSF:
\begin{equation}
	b(\lambda)=\int_{4\pi} \beta(\psi,\lambda)d\Omega=2\pi\int_0^\pi \beta(\psi,\lambda)sin\psi d\psi
\end{equation}
And \textbf{backscatter coefficient}:
\begin{equation}
	b_b(\lambda)\equiv 2\pi\int_{\pi/2}^\pi \beta(\psi,\lambda)sin\psi d\psi
\end{equation}


\begin{figure}[H]
\centering
  		\includegraphics[height=4cm]{/Users/javier/Desktop/Javier/PHD_RIT/20123_Spring/Modeling/HydroLight/Beamer/VSFweb.png}
\end{figure}
Example VSFs:\\
\textcolor{blue}{blue curve} : clear, open ocean water ($\lambda = 514 nm$)\\
\textcolor{green}{green curve} : harbor ($\lambda = 514 nm$)\\
\textcolor{red}{red curve} : very productive coastal waters ($\lambda = 530 nm$)

\textbf{Backscatter fraction:}
\begin{equation}
	B_b=\frac{b_b}{b}
\end{equation}
\textbf{Single-scattering albedo:}
\begin{equation}
	\omega_o=\frac{b(\lambda)}{c(\lambda)}
\end{equation}
\textbf{Volumen scattering phase function:}
\begin{equation}
	\tilde{\beta}(\psi,\lambda)\equiv \frac{\beta(\psi,\lambda)}{b(\lambda)}~~\left[sr^{-1} \right]\Rightarrow \beta(\psi,\lambda)=b(\lambda)\tilde{\beta}(\psi,\lambda)
\end{equation}

{Apparent Optical Properties}
\textbf{AOPs:} depend both on the medium and on the directional structure of the ambient light field

\vspace{\baselineskip}

\textbf{Irradiance reflectance:}
\begin{equation}
	R(z,\lambda)\equiv \frac{E_u(z,\lambda)}{E_d(z,\lambda)}
\end{equation}
\textbf{Remote sensing reflectance:}
\begin{equation}
	R_{rs}(\theta,\varphi,\lambda)\equiv \frac{L_w(\theta,\varphi,\lambda)}{E_d(\lambda)}~~\left[sr^{-1} \right]
\end{equation}
where $L_w$ is the \textbf{water-leaving radiance}

Under typical oceanic conditions, irradiances and radiances decrease exponentially with depth, then
\begin{equation}
	E_d(z,\lambda)\equiv E_d(0,\lambda) exp\left[-\int_0^{z}K_d(z',\lambda)dz'\right]
\end{equation}
where $K_d(z,\lambda)$ is the \textbf{spectral diffuse attenuation coefficient} for spectral downwelling plane irradiance. 
\begin{align}
	K_d(z,\lambda) 	&=- \frac{d\ln E_d(z,\lambda)}{dz} \notag \\
					&=-\frac{1}{E_d(z,\lambda)}\frac{dE_d(z,\lambda)}{dz} ~~\left[m^{-1} \right]
\end{align}

\textbf{Note:} IOPs are additive whereas AOPs are not

{Optical Constituents of Water}
\begin{itemize}
	\item Water
	\item Dissolved Organic Compounds (CDOM)
	\item Organic Particles (Chlorophyll)
	\item Inorganic Particles (Suspended Matter)
\end{itemize}

\textbf{Total IOP:}

\begin{equation}
	a_{total}(z,\lambda) = \sum_{i=1}^{ncomp} a_i(z,\lambda)
\end{equation}
e.g.,
\begin{equation}
	a_{total}(z,\lambda) =  a_w(\lambda) +a_{CDOM}(z,\lambda)+ a_{Chl}(z,\lambda)+a_{SM}(z,\lambda) \notag
\end{equation}

{The Radiative Transfer Equation}
\textbf{RTE} expresses conservation of energy for a collimated beam of radiance traveling through an absorbing, scattering and emitting medium

\vspace{\baselineskip}

The \textbf{RTE}:
\begin{align}
	\cos\theta\frac{dL(z,\theta,\varphi,\lambda)}{dz}&=-c(z,\lambda)L(z,\theta,\varphi,\lambda)\cdots \notag \\
	&+\int_{4\pi} L(z,\theta',\varphi',\lambda)\beta(z;\theta',\varphi' \to \theta,\varphi;\lambda)d\Omega'\cdots \notag  \\
	&+S(z,\theta,\varphi,\lambda)~~\left[W~m^{-3}sr^{-1}nm^{-1} \right]
\end{align}

\begin{figure}[H]
\centering
  		\includegraphics[height=4cm]{/Users/javier/Desktop/Javier/PHD_RIT/20123_Spring/Modeling/HydroLight/Beamer/RTE1.png}
\end{figure}

\begin{equation}
	-c(z,\lambda)L(z,\theta,\varphi,\lambda)\cdots \notag 
\end{equation}

\begin{figure}[H]
\centering
  		\includegraphics[height=4cm]{/Users/javier/Desktop/Javier/PHD_RIT/20123_Spring/Modeling/HydroLight/Beamer/RTE2.png}
\end{figure}

\begin{equation}
+\int_{4\pi} L(z,\theta',\varphi',\lambda)\beta(z;\theta',\varphi' \to \theta,\varphi;\lambda)d\Omega'\cdots \notag
\end{equation}

{\centering
\begin{equation}
	S(z,\theta,\varphi,\lambda) \notag
\end{equation}}

\begin{itemize}

\item {$S$ account for bioluminescence or inelastic-scattering processes}

\vspace{\baselineskip}

\item For example, the model is first run at the wavelength of excitation, $\lambda_{ex}$ to compute the energy shifted by inelastic scattering from $\lambda_{ex}$ to another wavelength $\lambda$, and the model is run again at $\lambda$, with the radiance shifted from $\lambda_{ex}$ appearing as a source term $S$ at $\lambda$

\end{itemize}

The \textbf{RTE}:
\begin{align}
	\cos\theta\frac{dL(z,\theta,\varphi,\lambda)}{dz}&=-c(z,\lambda)L(z,\theta,\varphi,\lambda)\cdots \notag \\
	&+\int_{4\pi} L(z,\theta',\varphi',\lambda)\beta(z;\theta',\varphi' \to \theta,\varphi;\lambda)d\Omega'\cdots \notag  \\
	&+S(z,\theta,\varphi,\lambda)~~\left[W~m^{-3}sr^{-1}nm^{-1} \right]
\end{align}
% ----------------------------------------------------------------------------------
\section{Water Quality Parameters Retrieval}
\cite{Jensen}
\cite{Mustard2001}
\subsection{Governing Equation}

\hl{The comparison is performed in the reflectance domain.}

\hl{CHL, SM and CDOM explanation}

From the total energy coming from the sun, only approximately 1390 $\left[\frac{W}{m^2}\right]$ reaches the Earth's atmosphere \cite{Schott}. This integrated value is known as the \emph{exoatmospheric irradiance}, or $E_S'$, and represents the total energy per unit area just outside the Earth's atmosphere due to the solar energy. Recall that \emph{irradiance} is the rate at which the radiant flux ($\Phi$) is delivered to a surface ($A$), defined as

\begin{equation} \label{eq:irradiance}
E = \frac{d\Phi}{dA}   \indent   \indent  \left[\frac{W}{m^2}\right]  
\end{equation} 

So, the $E_S'$ is calculated assuming that the flux $\Phi$ comes from a point source at the center of the sun such that it would produce an existence at the sun's surface, producing a flux at the mean Earth-sun distance of 1390 $\left[\frac{W}{m^2}\right]$ . For the present work, it is more convenient to express the irradiance spectrally, or other words as a function of wavelength, so we can describe the energy at desired wavelength, or spectral band.

The light source, usually the sun, interact with the target and then reach the sensor. This interaction will help us to extract about the target, in this case the water body. That is why in order to understand how the water quality parameter are retrieved, first it is necessary to introduce the concept of sensor reaching radiance. The sensor reaching radiance is defined as the accumulation of photons at the front of a sensor that one wishes to collect in an effort to obtain information about the target \cite{Gerace}. The total sensor-reaching radiance is the sum of the radiances due to the individual solar and thermal paths.  \cite{Schott} shows that in the VNIR region (approximately 0.3-2.5 [$\mu m$]), the solar energy is so many orders of the magnitude higher than the self-emitted energy, so the thermal paths are negligible for this study. Also, we consider that radiance from the background is negligible because the water bodies are typically several kilometers wide. Assuming that the target is approximately Lambertian (radiance is equal in all directions), the total sensor-reaching radiance, $L$, is defined as

\begin{equation} \label{eq:gov1}s
L(\lambda) = \frac{E'_S(\lambda)cos(\sigma')r(\lambda)\tau_1(\lambda)\tau_2(\lambda)}{\pi} +
                        \frac{E_{ds}(\lambda)r(\lambda)\tau_2(\lambda)}{\pi} + L_{us}(\lambda)
\end{equation} 
where:
\begin{tabbing}
\indent \indent \indent  $L(\lambda)$ \hspace{1mm}\=:  \indent \= total sensor-reaching radiance\\
\indent \indent \indent  $E'_S(\lambda)$\>: \>exoatmospheric spectral irradiance\\
\indent \indent \indent $\sigma'$\>:\>solar-zenith angle\\
\indent \indent \indent $r(\lambda)$\>:\>spectral reflectance target\\
\indent \indent \indent $\tau_1(\lambda)$\>:\>Sun-target path transmission of atmosphere\\
\indent \indent \indent $\tau_2(\lambda)$\>:\>target-sensor path transmission of atmosphere\\
\indent \indent \indent $E_{ds}(\lambda)$\>:\>solar downwelling irradiance\\
\indent \indent \indent $L_{u}(\lambda)$\>:\>solar upwelling irradiance\\
\end{tabbing}

Equation \eqref{eq:gov1} is the solar term of the "big equation" described by \cite{Schott}.

\subsection{Constituent Retrieval}

When imaging water body, a set of new paths need to be taken in account to determine the sensor-reaching radiance.
 \begin{figure}[H]
	\centering
    	\includegraphics[width=100mm]{/Users/javier/Desktop/Javier/MASTER_RIT/2011_THESIS/LaTeX/Thesis/Images/WaterColumn.eps}
 	\caption{Contributions to sensor-reaching radiance from the water column \label{fig:WaterColumn}}
  \end{figure}


Each sensor-reaching radiance curve is associated with a specific combination of water components (CHL, SM and CDOM). HidroLight provides us the \hl{remote sensing reflectance}, $R_{RS}$, which describes how much of the total incident downwelling irradiance is ultimately returned from a water column in a given direction, defined as

\begin{equation} \label{eq:Rrs}
R_{RS}(\theta,\phi,\lambda,z=a) = \frac{L(\theta,\phi,\lambda,z=a)}{E_d(\lambda,z=a)}   \indent   \indent  \left[\frac{1}{sr}\right]  
\end{equation} 
where:
\begin{tabbing}
\indent \indent \indent  $\theta$ \hspace{1.5mm}\=:  \indent \= sensor-zenith angle\\
\indent \indent \indent  $\phi$\>: \>sensor-azimuth angle\\
\indent \indent \indent $L$\>:\>water-leaving radiance\\
\indent \indent \indent $E_d$\>:\>total downwelling irradiance\\
\indent \indent \indent $\lambda$\>:\>wavelength dependent\\
\indent \indent \indent $a$\>:\>height just above the water's surface\\
\end{tabbing}



\section{Shallow Water Retrieval}
\section{LUT Method}
\subsection{Population}
\subsection{Optimization Algorithm  - lsqnonlin}

\section{State of the Research}

As shown in \cite{Gerace}
and \cite{Mobley} and \cite{Lesser}

\section{Atmospheric Compensation}
% !TEX root=ProposalJavier.tex 
% the previous is to reference main .bib
%% CHAPTER
\chapter{Methodology and Approach}
\label{ch:method}

The methodology is separated into the specific objectives mentioned in Section \ref{ch:objectives}. In this work, a look-up-table (LUT) methodology was implemented to retrieve concentration of water constituents using Landsat 8 imagery. Figure~\ref{fig:retrieval} shows a diagram of this retrieval process. First, the Landsat 8 image data (shown at the top of the figure) needs to be atmospherically corrected. Then, a non-linear optimization routine uses the water pixels (reflectance values) and a LUT of reflectance spectra to estimate concentrations for each water pixels in the scene. The outputs to the process are concentration maps for each water constituents, as shown in Figure~\ref{fig:retrieval}. This process is explained in details in Subsections \ref{sec:atmcorr} and \ref{sec:retrieval} below.
\begin{figure}[htb]
  \centering
  \includegraphics[height=7cm]{/Users/javier/Desktop/Javier/PHD_RIT/ConferencesAndApplications/NESSF14/latex/Retrieval.pdf}
  \caption{Retrieval process diagram. \label{fig:retrieval} } 
\end{figure}

\section{Atmospheric Correction} 
\label{sec:atmcorr}
The first objective in this research is to identify a suitable approach to atmospherically correct the type of dataset provided by the OLI sensor. This is a complex task to perform over water because the signal leaving the water that reaches the sensor is very small compared to the signal reaching the sensor from atmospheric scattering. Most of the atmospheric correction algorithms applied to ocean color satellites are not suitable for highly turbid coastal waters because the {\it black pixel assumption} cannot be applied to these types of waters~\cite{Patt2003}. Two methods are be investigated in this research.

\subsection{Model-based ELM}
The first method will be a model-based empirical line method (ELM) based on previous work done by \cite{Gerace:2013} and \cite{Gerace:2012}  for simulated OLI data. While this new method is based on the traditional ELM method (see Section \ref{subsec:ELM}), this model-based ELM method tries to avoid the measurement of ground truth at every sensor passover over the scene by using pseudo-invariant features (PIF) \index{pseudo-invariant features (PIF)} in the scene as one target along with a estimation of water reflectivity for an open lake region for the other target. The two targets used in this model-based ELM to solve the regression in \autoref{eq:ELM} are referred to in this documents as the {\it bright pixel} \index{bright pixel} and the {\it dark pixel}\index{dark pixel}.

This method employs a PIF pixel extraction \cite{Schott:1988} to mask urban landscape from both the reflectance product and the L8 image for the bright pixel determination. Pseudo-invariant targets are defined as targets whose reflectivity properties do not change rapidly between different times of collection. Examples of pseudo-invariant target are urban features in the scene.  The PIF extraction isolates the pseudoinvariant features from the digital imagery. In our case, the PIF are the man-made urban features in a scene. A flowchart of the process is shown in \autoref{fig:PIFflowchart}. 

\begin{figure}[htb]
	\centering
  \begin{tikzpicture}[node distance=0.75cm, auto]
          \tikzset{
                  basenode/.style={rectangle,rounded corners,draw=black,very thick, inner sep=1em, minimum size=3em, text centered,text width=2cm},
                  productnode/.style={ellipse,rounded corners,draw=black, very thick, text centered,text width=1.5cm},
                  myarrow/.style={->,>=stealth',thick, double = black},
                  mylabel/.style={text width=7em, text centered}
          }
          % SWIR branch
          \node[basenode] (SWIR) {SWIR 2\\ Band};
          \node[basenode, below=of SWIR] (TS1) {Mask by Threshold (upward)};
          \node[align=left, right=0.0 of TS1] (C1) {Urban\\Veget.\\Water};
          \node[align=left, right=-0.15 of C1] (C2) {ON\\ON\\OFF};

          % Ratio branch
          \node[basenode, right=2.5cm of SWIR] (Ratio) {Ratio\\ NIR Band/ Red Band};
          \node[basenode, below=of Ratio] (TS2) {Mask by Threshold (downward)};
          \node[align=left, right=0.0 of TS2] (C3) {Urban\\Veget.\\Water};
          \node[align=left, right=-0.15 of C3] (C4) {ON\\OFF\\ON};

          % AND
          \path (TS1.south)--(TS2.south) node[pos=.5,below=2cm] (AND) {.AND.};


          % PIF Mask
          \node[basenode, below=of AND] (PIFMask){PIF Mask};
          \node[align=left, left=0.85 of PIFMask] (C5) {Urban\\Veget.\\Water};
          \node[align=left, right=-0.15 of C5] (C6) {ON\\OFF\\OFF};

          \node[basenode, below=of TS2,right=2.0cm of AND] (Image) {Image};
          \path (Image.south)--(PIFMask.east) node[below=of Image,right=2cm of PIFMask] (AND2) {.AND.};
          \node[basenode, right=2cm of AND2] (PIFIm){PIF Image};

          \draw[myarrow] (SWIR)--(TS1);
          \draw[myarrow] (Ratio)--(TS2);
          \draw[myarrow] (TS1)--(AND);
          \draw[myarrow] (TS2)--(AND);
          \draw[myarrow] (AND)--(PIFMask);
          \draw[myarrow] (Image)--(AND2);
          \draw[myarrow] (PIFMask)--(AND2);
          \draw[myarrow] (AND2)--(PIFIm);
  \end{tikzpicture}
\caption{Illustration of the logic used to segment PIF features. \label{fig:PIFflowchart}}
\end{figure}

The PIF extraction from digital imagery proceeds in the following fashion. An infrared-to-red ratio image is very effective in the classification of water, vegetation, and urban features. The vegetation in this ratio image will tend to have a high brightness when compared to the urban features and water brightness. This infrared-to-red ratio image can be derived from the quotient of the NIR band (band 4 for Landsat-5; band 5 for Landsat-8) and the red band (band 3 for Landsat-5,band 4 for Landsat-8), as seen in \autoref{fig:PIFflowchart}. This ratio image is thresholded from the high digital count values downward to create a mask so the high brightness pixels are eliminated (vegetation pixels) from the image, that is, these pixels are set to a value of zero and the rest (water and urban pixels) to a value of one. The SWIR 2 band (band 7 in Landsat-5 and Landsat-8) is used to eliminate the water pixels from the previous mask since water has nearly zero reflectance in this spectral region. This SWIR 2 band is thresholded from the low brightness values upward. Water pixels will exhibit a low value when compare to the rest of the pixels. A mask is created by assigning a value of zero to the low brightness pixels (water pixels) and a value of one to the rest (urban features and vegetation). Finally, the two mask created are combined using a logical .AND. resulting in a mask that will have a value of one only in the urban feature pixels, i.e. the PIFs, as shown in \autoref{fig:PIFflowchart}. This mask will be named PIF mask for the rest of this document. An example of a PIF mask is illustrated in \autoref{fig:PIFmask}. A false color image of the Downtown Rochester, NY is shown on the left (vegetation in red) and a RGB image of the same area with the PIF mask applied is shown on the right (urban features in bright color while masked pixels in black).

% \vspace{-.3cm}
\begin{figure}[htb]
  \begin{minipage}[c]{0.48\linewidth}
    \centering
      \includegraphics[trim=30 0 30 0,clip,height=6cm]{/Users/javier/Desktop/Javier/PHD_RIT/Latex/Proposal/Images/DTROCL8falsecolor.jpg}  
    % \vspace{1.5cm}
    \centerline{(a)}\medskip
  \end{minipage}
  \hfill
  \begin{minipage}[d]{0.48\linewidth}
    \centering
      \includegraphics[trim=30 0 30 0,clip,height=6cm]{/Users/javier/Desktop/Javier/PHD_RIT/Latex/Proposal/Images/PIFmaskApplied.jpg}
    % \vspace{1.5cm}
    \centerline{(b)}\medskip
  \end{minipage}
  \caption{PIF mask determination. (a) False color image, with vegetation in red and (b) PIF mask over downtown Rochester. \label{fig:PIFmask} } 
\end{figure}

The PIF mask is used to determine the bright pixel spectra in both radiance (from the L8 image) and reflectance values (from the Landsat Surface Reflectance CDR \cite{LandsatCDR} image). See Section~\ref{sec:CDR} for more details about the Landsat reflectance product. The Landsat reflectance product was available for a total of 9 Landsat 5 scenes where clear sky conditions were acceptable. One PIF mask for each of these 9 scenes was created using ENVI. In addition, one PIF mask was created from the L8 radiance image. Finally, these 10 PIF masks were combined to create a ``master'' PIF mask \index{master PIF mask}. Then, the statistics were calculated in ENVI for each scene using this master PIF mask. An example of the statistical results obtained from ENVI are shown in \autoref{fig:PIFstats}.(a) and \autoref{fig:PIFstats}.(b) for one scene of the Landsat reflectance product (in reflectance units) and for the L8 image (in radiance units), respectively. The mean value is shown in black solid line, the green solid lines are the mean plus standard deviation and the mean minus standard deviation, and the red solid lines are the maximum and minimum values for each band. The mean values for each one of the 9 scenes are shown in \autoref{fig:ZenithCorr}. 

\begin{figure}[h!]
  \begin{minipage}[c]{0.48\linewidth}
    \centering
      \includegraphics[height=9cm]{/Users/javier/Desktop/Javier/PHD_RIT/Latex/Proposal/Images/PIFstatCDR.png}  
    % \vspace{1.5cm}
    \centerline{(a)}\medskip
  \end{minipage}
  \hfill
  \begin{minipage}[d]{0.48\linewidth}
    \centering
      \includegraphics[height=9cm]{/Users/javier/Desktop/Javier/PHD_RIT/Latex/Proposal/Images/PIFstatL8.png}
    % \vspace{1.5cm} 
    \centerline{(b)}\medskip
  \end{minipage}
  \caption{Bright pixel determination using the PIF mask in ENVI. Statistics with the PIF mask applied for (a) Landsat reflectance product (in reflectance units) and (b) statistics for L8 image (in radiances units). \label{fig:PIFstats} } 
\end{figure}

As seen in \autoref{fig:ZenithCorr}, the PIF reflectance values for each scene are not the same, but a high correlation between the reflectance values and the solar zenith angle for each band was found. A linear relationship was determined for each band by applying a linear regression in MATLAB and the $R^2$ and root mean square error (RMSE) values were calculated as a way to measure this correlation. This linear relationship has the form 
\begin{equation}
	y = m*x + y_0
	\label{eq:linear}
\end{equation}
where $x$ represents the solar zenith angle and $m$ the reflectance value. \autoref{fig:Band1Corr} shows the reflectance values versus the solar zenith angle for band 1 for the 9 Landsat reflectance scenes and the calculated linear relationship. The values $m$ and $y_0$ found for all the bands are shown in \autoref{tab:ZenithCorr} along with the $R^2$ and RMSE values for each band. It can been seen in \autoref{tab:ZenithCorr} the $R^2$ values are bigger than $90\%$ for all bands, which suggests there are a high correlation between the reflectance values and the solar zenith angle. As a conclusion, these results show that the reflectance values remain constant over time and depend of the solar zenith angle of the sensor. 

The L8 image has associated a particular solar zenith angle. The previous linear relationships calculated will help to estimate the values for the reflectance value of the bright pixel for that particular solar zenith angle. For example, the solar zenith angle for the 09-19-13 L8 scene is equal to $45^\circ$, and therefore $x=45^\circ$ in \autoref{eq:linear}. The reflectance values for $x=45^\circ$ are shown in the last column of \autoref{tab:ZenithCorr} and plotted in \autoref{fig:ZenithCorr} in red asterisks.

Because at the moment of writing this documents, the Landsat reflectance products was not available for L8, it was necessary to estimate an theoretical reflectance value for the coastal band for Landsat 5 in order to match the L8 bands. To accomplish this, it was assumed that the coastal band would exhibit a similar trend than the blue and green bands. Hence, a straight-line that passes over the blue and green band values was used to extrapolate the value of the coastal band. 
%--------------------------------------
\vspace{.5cm}
\begin{table}[h!]
\caption{ Zenith angle correction parameters. \label{tab:ZenithCorr} } 
\centering
\begin{tabular}{c|c|c|c|c|c} 
 \bfseries{Band n} & \bfseries{$m$}      & \bfseries{$y_0$}    & \bfseries{$R^2$}     & \bfseries{$RMSE$} & $y(x=45^\circ)$   \\ \hline \hline
 Band 1 & -0.000412 & 0.122631 & 0.937155 & 0.001705 &  0.1041\\
 Band 2 & -0.000634 & 0.147424 & 0.934344 & 0.002685 &  0.1189\\
 Band 3 & -0.000756 & 0.161421 & 0.976599 & 0.001869 &  0.1274\\
 Band 4 & -0.001316 & 0.220031 & 0.906946 & 0.006733 &  0.1608\\
 Band 5 & -0.001148 & 0.217231 & 0.903702 & 0.005984 &  0.1656\\
 Band 6 & -0.001159 & 0.206725 & 0.929626 & 0.005096 &  0.1546\\  
 \end{tabular}
\end{table}

\begin{figure}[htb]
  	\centering
  	\includegraphics[height=7cm]{/Users/javier/Desktop/Javier/PHD_RIT/Latex/Proposal/Images/ZenithCorrelation.eps}
  \caption{Correlation for band 1. \label{fig:Band1Corr} } 
\end{figure}

\begin{figure}[htb]
  	\centering
  	\includegraphics[height=7cm]{/Users/javier/Desktop/Javier/PHD_RIT/Latex/Proposal/Images/ZenithCorrection.eps}
  \caption{Bright pixel for 9 different scenes. \label{fig:ZenithCorr} } 
\end{figure}

The radiance spectra for the dark pixel is obtained from a ROI in the water present in the L8 image that could be considered a dark region (i.e. open lake). Statistics are computed in this dark region, and the mean values in each band is used as radiance spectra for the dark pixel. The reflectance spectra for the dark pixel is obtained from a HydroLight run for conditions  representative of the same ROI in the water present in the L8 scene. Some conditions are the concentrations of water constituents, solar zenith angle, IOPs, backscatter phase function, for example. In order to have a reflectance spectra from HydroLight representative of the area of study, IOPs corresponding to the particular ROI need to be input into HydroLight. The measured IOPs and concentration measurements from ground truth collection are ideally used. Then, the reflectance spectra for the dark pixel from HydroLight is spectrally sampled to the L8 sensor response. \todo{mentions an example of concentrations}

\missingfigure{figure showing and ROI in the L8 image and statistic in ENVI}
\missingfigure{figure showing the HL run}
\missingfigure{figure showing the IOPs}

As a resume, \autoref{fig:ELMpxsENVI} shows the different spectra used to perform the model-based ELM, where four different spectra can be seen: one reflectance and one radiance spectra for the bright pixel (obtained using the PIF extraction over the Landsat reflectance product and L8 image, respectively), one reflectance spectra for the dark pixel (obtained from HydroLight), and one radiance spectra for bright pixel (obtained from the statistics of a ROI over water in the L8 image). These spectra are used to atmospherically correct the L8 image using the ENVI Classic software \cite{ENVIUserGuide}. This is performed using the ``Empirical Line'' algorithm of the ``Calibration Utilities'' in ENVI classic, where the L8 image is used as the input image, and the reflectance spectra are labeled as ``field spectra'' and the radiance spectra are labeled as ``data spectra.'' The product of this process is an image in reflectance values, which will be used to perform the retrieval of water constituents described in Section \ref{sec:retrieval} below. 

\begin{figure}[htb]
  \centering
  \includegraphics[width=14cm]{/Users/javier/Desktop/Javier/PHD_RIT/Latex/Proposal/Images/ELMpixelsENVI.pdf}
  \caption{Bright and Dark pixels used in ENVI to apply ELM. \label{fig:ELMpxsENVI} } 
\end{figure}

Preliminary results of the model-based ELM atmospheric correction methods for an area of water in the Rochester, NY region are shown in Figure~\ref{fig:waterpxs}. This figure shows the spectrum of the water pixels after atmospheric correction (in reflectance values), and these exhibit shapes that correspond with the shapes of typical water pixels. Figure~\ref{fig:refcomp} shows water surface reflectance spectra as preliminary results from the model-based ELM method (solid lines) compared with results from a traditional ELM method (dashed lines) for four different regions of interest (ROIs) in the Rochester area (Cranberry Pond, Long Pond, and nearshore and offshore of the Lake Ontario). As can be seen, the atmospheric correction algorithm proposed exhibits less than one reflectance unit of difference in comparison to the results from the traditional ELM algorithm. While small, these differences have a significant impact on the retrieved water constituents. Therefore, the proposed atmospheric correction algorithm will be improved by this research. A further validation with ground-truth data is anticipated at the end of this research.
% \vspace{-.3cm}
\begin{figure}[htb]
  \begin{minipage}[c]{0.48\linewidth}
    \centering
      \includegraphics[height=6cm]{/Users/javier/Desktop/Javier/PHD_RIT/ConferencesAndApplications/NESSF14/latex/WaterPixels_2.eps}
      \caption{Water pixel spectra after applying the model-based ELM atmospheric correction method.}
      \label{fig:waterpxs}
    % \vspace{1.5cm}
    % \centerline{(a)}\medskip
  \end{minipage}
  \hfill
  \begin{minipage}[d]{0.48\linewidth}
    \centering
      \includegraphics[height=6cm]{/Users/javier/Desktop/Javier/PHD_RIT/ConferencesAndApplications/NESSF14/latex/WaterPixelComparisonELMELMbased}
      \caption{Comparison between traditional ELM (dashed lines) and model-based ELM (solid lines).}
      \label{fig:refcomp}
    % \vspace{1.5cm}
    % \centerline{(b)}\medskip
  \end{minipage}
  % \caption{Water pixel spectra: (a) after atmospheric correction and (b) comparison with traditional ELM method.}
\end{figure}


\subsection{Gordon's method for turbid water}
The second method will be an extension of a method developed for SeaWiFS over turbid coastal and inland waters \cite{Ruddick:2000bs}. This method is a modified version of the methods developed by Gordon \cite{Gordon:1997} for ocean color satellites, but when the signal leaving the water does contribute to the overall signal beyond the NIR part of the spectrum. By using longer wavelengths and restricting the input pixels to open waters, these methods can be  applied to many fresh and coastal regions. The water surface reflectance values obtained after atmospheric correction will be validated through comparison to water surface reflectance measured in situ. 
% -----------------------------------------------
\section{In-Water Constituent Retrieval Process}
\label{sec:retrieval}
\todo{Mention: water mask}The retrieval algorithm will be based on previous work done by Gerace {\it et al.} \cite{Gerace:2013} and Raqueno {\it et al.} \cite{Raqueno:2000}. The water surface reflectance product obtained after atmospheric correction from the previous stage is used as input to the retrieval algorithm. Each pixel in the reflectance product has an unknown concentration. A spectral matching technique is applied to predict this concentration by comparing the spectral shape of each pixel with the elements in a look-up table (LUT). The LUT is generated in HydroLight \cite{MobleyHE} for different triplets of water constituent concentrations. The spectral matching is made by a least square error minimization algorithm using the ``lsqnonlin'' package of the MATLAB's Optimization Toolbox. The output of this process is a concentration mapping for each water constituent that spans the range of constituents levels in the scene. An example of a LUT created in HydroLight is shown in Figure~\ref{fig:results1}. Preliminary results for concentration maps on a logarithmic scale over the area of study are shown in Figure~\ref{fig:results2} (Chlorophyll-{\it a} (CHL) in $[\mu g/L]$, SM in $[mg/L]$, and CDOM in $[1/m]$).

\begin{figure}[htb]
  \begin{minipage}[c]{0.48\linewidth}
    \centering
      \includegraphics[height=6cm]{/Users/javier/Desktop/Javier/PHD_RIT/ConferencesAndApplications/NESSF14/latex/LUTPixels_2.eps}
      \caption{LUT created in HydroLight}
      \label{fig:results1}
    % \vspace{1.5cm}
    % \centerline{(a)}\medskip
  \end{minipage}
  \hfill
  \begin{minipage}[d]{0.5\linewidth}
    \centering
      \includegraphics[height=5cm]{/Users/javier/Desktop/Javier/PHD_RIT/ConferencesAndApplications/NESSF14/latex/RetrievalResultLog_2.eps}
      \caption{Concentration mapping (Log scale).}
      \label{fig:results2}
    % \vspace{1.5cm}
    % \centerline{(b)}\medskip
  \end{minipage}
  %
  % \caption{Preliminary results: (a) LUT created in HydroLight and (b) concentration maps.}
  % \label{fig:results}
\end{figure}

\section{Ground Truth Data Collection}
The area of study for this research is the Lake Ontario Rochester Embayment (latitude: 43°15'32.53"N and longitude: 77°36'13.10"W), which includes some nearby ponds (Long and Cranberry Ponds), the Genesee River plume, the Irondequoit Bay and the southern end of Lake Ontario, as shown in Figure~\ref{fig:areaofstudy1} and Figure~\ref{fig:areaofstudy2}. This area was selected because it exhibits a wide range of variability in concentration of water constituents, so the retrieval algorithm can be tested with different scenarios. Landsat 8 images from this area of study and corresponding water samples collected at the time of the satellite's overpass will be used to test the retrieval algorithm. So far, there are only three satisfactory images available from the summer 2013. This project contemplates performing one new ground truth data collection during 2014. Therefore, images from the 2013-2014 spring and summer collection seasons will be used to test the methodology. Note that a difficult challenge of this research is to obtain images with relatively clear weather conditions (i.e. cloud free) over the area of study.
\begin{figure}[htb]
  \centering
  \includegraphics[height=9cm]{/Users/javier/Desktop/Javier/PHD_RIT/Latex/Proposal/Images/AreaOfStudy1.pdf}
  \caption{Area of Study. \label{fig:areaofstudy1} } 
\end{figure}
\begin{figure}[htb]
  \centering
  \includegraphics[height=7cm]{/Users/javier/Desktop/Javier/PHD_RIT/Latex/Proposal/Images/AreaOfStudy2.pdf}
  \caption{Area of Study. \label{fig:areaofstudy2} } 
\end{figure}

In order to have outputs in HydroLight that are representative of the water bodies that are being studied, inherent optical properties (IOPs) of those specific waters have to be defined as input to the HydroLight model. After collection, these water samples need to be analyzed in the lab to obtain IOPs for the main water constituents. Furthermore, apparent optical properties (AOPs) (i.e. water surface reflectance) and backscattering measurements will be also collected for further comparison and to pursue closure between the HydroLight AOPs results and in-situ AOPs measurements.

\section{Validation}
The results from the retrieval process will be validated by comparison with the concentration of water samples taken during field campaigns in the spring and summer of 2013, 2014 and 2015. These concentrations will be obtained from lab measurements made at the Rochester Institute of Technology. For further validation, the results will be compared with products derived from ocean color satellites such as MODIS (e.g. MODIS Chl-{\it a} product), in regions where it is possible.
% \section{OLI Sensor}
% \subsection{Sensor Response}
% Noise level. Quantization. SNR.

% \section{Retrieval Process}
% \subsection{The Look-Up Table}


% \subsubsection{Real Atmosphere Conditions}

% \section{Over-Water Atmospheric Compensation}
% \subsection{Model Based Empirical Line Method}

% \subsubsection{Pseudo Invariant Features}

% This model employs pseudo-invariant feature (PIF) pixels extraction from the Landsat Climate Data Record (CDR) Surface Reflectance product along with an in-water radiative transfer model (HydroLight) to obtain the field spectra to be used in the ELM method. 
%  A mask is created applying a threshold a the ratio between the NIR band and the red band. 
%  A mask is created applying a threshold to the SWIR 2 band. 

% The model based empirical line method (MBELM) atmospheric correction method is based in the well known empirical line method (ELM).

% \subsection{LUT Method}

\section{Concluding Remarks}
% !TEX root=ProposalJavier.tex 
% the previous is to reference main .bib
%% CHAPTER
\chapter{Initial Results}
\label{ch:results}
This chapter is separated in four sections. The first section explains the data collection process, giving some details in the kind of measurements taken in the field. Also, it includes a summary of the data collected at the moment of writing this document. The second section gives a brief description of the lab measurements. The atmospheric section shows preliminary results from the model-based ELM atmospheric correction method. Finally, the last section shows preliminary retrieval results over the area of study.
% ------------------------------
\section{Data Collection}

The first area of study used in this research is the Rochester Embayment in the city of Rochester, NY. This area of study includes some nearby ponds (Long and Cranberry ponds), the Genesee River plume, the Irondequoit bay and part of Lake Ontario. This area was selected because it exhibits a wide range of variability in concentration of water constituents, so the retrieval algorithm can be tested with different scenarios. \autoref{fig:0910913Sites} shows an image over this area of study for the data collection done on September, $19^{th}$, 2013  with the different sites as an example. The data collections are divided in two crews. One crew, named ``Lake crew'', is in charge of the Irondequoit Bay, Ontario near shore, Ontario off shore, Genesee River plume, Genesee River pier sites (labeled in \autoref{fig:0910913Sites} as IBayN, OntNS, OntOS, RvrPLM and RvrPIER, respectively). The other crew, named ``Ponds crew'', is in charge of the Long Pond north and south, Cranberry Pond sites (labeled in \autoref{fig:0910913Sites} as LongN, LongS and Cranb, respectively).

\begin{figure}[htb]
  \centering
  \includegraphics[width=12cm]{/Users/javier/Desktop/Javier/PHD_RIT/Latex/Proposal/Images/groundtruth-sitenames-no-ends.jpg}
  \caption{Sites in the Rochester Embayment for the water sample collection on September, $19^{th}$, 2013.\label{fig:0910913Sites} } 
  % \vspace{1cm}
\end{figure}

Water samples are collected for each site in dark Nalgene bottles. Additionally, remote-sensing reflectances are measured using an ASD and a SVC instrument (one for each crew). Backscattering measurements are taken using a HydroScat-2. This measurement is taken by both crew only if the logistic of the particular day allows it. Otherwise priority is given to the Lake crew. For each site, a location is recorded using a GPS. \autoref{tab:collect} shows a summary of the data collections done in 2013 and 2014 seasons with the respective available data.

\begin{table}[htb]
  \caption{Summary of 2013 and 2014 data collections.}
  \centering
  \includegraphics[width=13cm]{/Users/javier/Desktop/Javier/PHD_RIT/Latex/Proposal/Images/Collect1314.png}
  \label{tab:collect}
\end{table}

% ------------------------------
\section{Laboratory Measurements}

After collection, the water samples are analyzed in laboratory. These lab measurements include concentration of CHL and TSS, absorption coefficients of CHL, TSS and CDOM. The Chlorophyll-{\it a} concentration measurements needed to be validated with measurements analyzed by a credible lab (Monroe County Environmental Laboratory). This comparison with this lab shows agreement between the measurements. \autoref{tab:collect} shows the measurements available. \todo{include references to procedures}

% ------------------------------
\section{Atmospheric Correction}

Preliminary results of the model-based ELM atmospheric correction methods for different water bodies in the Rochester Embayment area are shown in \autoref{fig:waterpxs}. This figure shows the spectrum of the water pixels in reflectance values after applying the model-based ELM atmospheric correction. These curves exhibit shapes that correspond with the shapes of typical water pixels. \autoref{fig:refcomp} shows water reflectance spectra as preliminary results from the model-based ELM method (solid lines) compared with results from a traditional ELM method (dashed lines) for four different ROIs in the Rochester Embayment area (Cranberry Pond, Long Pond, and nearshore and offshore of the Lake Ontario, labeled as Cranb, LongS, OntNS and OntOS in \autoref{fig:0910913Sites}, respectively). 

The traditional ELM method was performed with reflectance measurements taken in the field. A reflectance measurement taken of the sand of Charlotte Beach, Rochester, NY (labeled as SandDry in \autoref{fig:0910913Sites}) was used for the bright pixel while a reflectance measurement taken at the site OntNS was used for the dark pixel. The radiance values were taken from the corresponding ROIs in the Landsat 8 image. 

As can be seen in \autoref{fig:refcomp}, the atmospheric correction algorithm proposed in this study exhibits less than one percent reflectance unit ($<0.01\Rightarrow <1\%$) of difference in comparison to the results from the traditional ELM algorithm.

\begin{figure}[htb]
  \begin{minipage}[c]{0.48\linewidth}
    \centering
      \includegraphics[height=6cm]{/Users/javier/Desktop/Javier/PHD_RIT/ConferencesAndApplications/NESSF14/latex/WaterPixels_2.eps}
      \caption{Water pixel spectra after applying the model-based ELM atmospheric correction method.}
      \label{fig:waterpxs}
    % \vspace{1.5cm}
    % \centerline{(a)}\medskip
  \end{minipage}
  \hfill
  \begin{minipage}[d]{0.48\linewidth}
    \centering
      \includegraphics[height=6cm]{/Users/javier/Desktop/Javier/PHD_RIT/ConferencesAndApplications/NESSF14/latex/WaterPixelComparisonELMELMbased}
      \caption{Comparison between traditional ELM (dashed lines) and model-based ELM (solid lines).}
      \label{fig:refcomp}
    % \vspace{1.5cm}
    % \centerline{(b)}\medskip
  \end{minipage}
  % \caption{Water pixel spectra: (a) after atmospheric correction and (b) comparison with traditional ELM method.}
\end{figure}


% ------------------------------
\section{Retrieval}

Preliminary results for concentration maps for each CPA over the Rochester Embayment, Rochester, NY are shown in Figure~\ref{fig:retrievalresults}. The expected trend of having low concentration of CPAs in the offshore of Lake Ontario and higher concentrations in the nearby ponds (Long Pond and Cranberry Pond) can be seen. 
\begin{figure}[htb]
\centering
\includegraphics[trim=200 100 180 0,clip,height=9cm]{/Users/javier/Desktop/Javier/PHD_RIT/ConferencesAndApplications/2014_RITResearchSymposium/Images/RetrievalResults.eps}
   \caption{Retrieval preliminary results.}
      \label{fig:retrievalresults}   
\end{figure}

Comparisons between retrieved CPAs concentrations and field measurements are shown in Figure~\ref{fig:chlcomp}, Figure~\ref{fig:tsscomp} and Figure~\ref{fig:cdomcomp} for four different stations in the area of study. This comparison with field measurements showed good agreement at low concentrations but differences at higher concentrations. Ongoing work is focusing on incorporation of the IOP differences between water bodies in the LUT optimization process.
\begin{figure}[htb]
\centering
    \includegraphics[height=7cm]{/Users/javier/Desktop/Javier/PHD_RIT/ConferencesAndApplications/IGARSS2014/paper/Images/chlcomp.eps} 
    \caption{Comparison between measured and retrieved chlorophyll {\it a} concentration.}
    \label{fig:chlcomp} 
\end{figure}     

\begin{figure}[htb]
\centering
    \includegraphics[height=7cm]{/Users/javier/Desktop/Javier/PHD_RIT/ConferencesAndApplications/IGARSS2014/paper/Images/tsscomp.eps}   
    \caption{Comparison between measured and retrieved TSS concentration.}
    \label{fig:tsscomp} 
\end{figure}  

\begin{figure}[htb]
\centering
    \includegraphics[height=7cm]{/Users/javier/Desktop/Javier/PHD_RIT/ConferencesAndApplications/IGARSS2014/paper/Images/cdomcomp.eps}    
    \caption{Comparison between measured and retrieved CDOM concentration.}
    \label{fig:cdomcomp} 
\end{figure}  

% ------------------------------
\todo{Concluding Remarks}
% \section{Concluding Remarks}
\chapter{Future Work}
% !TEX root=ProposalJavier.tex 
% the previous is to reference main .bib
%% CHAPTER
\appendix
\chapter*{Appendix}
\addcontentsline{toc}{chapter}{Appendix}
\section{Lab Measurements}
\section{Field Measurements}
\section{Landsat-8 Sensor Characteristics}
\section{HydroLight Specification}
\section{MODTRAN Specification}

\listoftodos

%\bibliographystyle{ieeetr}
%\bibliographystyle{unsrtnat}
\bibliographystyle{apalike}

\bibliography{/Users/javier/Desktop/Javier/PHD_RIT/Latex/javier_bib}

\printindex

\end{document}  