\documentclass{book}

\usepackage{graphicx}
\usepackage{float}
\usepackage{amssymb,amsmath}
\newcommand{\bm}[1]{\boldsymbol{#1}}
\usepackage{indentfirst}
\setlength{\parindent}{1cm}
\usepackage{multirow}
\usepackage{fullpage}
\usepackage{appendix}
\usepackage{setspace}
\usepackage[bf, small, center]{caption}
\setlength{\belowcaptionskip}{10pt}
\usepackage{longtable}

\usepackage{geometry}
\geometry{
%  top=1.0in,  
%  inner=1.5in,
%  outer=1.5in,
  bottom=1.2in,
%  headheight=3ex, 
  headsep=3ex,         
}

\usepackage{fancyhdr}
\pagestyle{fancy}
\fancyhead[LO, RE]{\rightmark}
\fancyhead[LE, RO]{\thepage}
\fancyfoot[]{}
%\renewcommand{\headrulewidth}{0.3pt}

\usepackage{array}
\newcolumntype{L}[1]{>{\raggedright\let\newline\\\arraybackslash\hspace{0pt}}m{#1}}
\newcolumntype{C}[1]{>{\centering\let\newline\\\arraybackslash\hspace{0pt}}m{#1}}
\newcolumntype{R}[1]{>{\raggedleft\let\newline\\\arraybackslash\hspace{0pt}}m{#1}}

\onehalfspacing

\usepackage[phd]{thesisfrontmatter}

% *** NOTE WHICH PROGRAM GENERATED IMAGES OR WHICH POWER POINT THEY ARE FROM
%compare methodology with script for final process
%how much detail in methodology?
%some sort of processing flowchart?
%how much do we care about processing?

\usepackage[parfill]{parskip}    % Activate to begin paragraphs with an empty line rather than an indent


\usepackage{mathtools}
\usepackage{epstopdf}

\usepackage{color}
\usepackage{soul}



\DeclareGraphicsRule{.tif}{png}{.png}{`convert #1 `dirname #1`/`basename #1 .tif`.png}

\let\stdsection\chapter  
\renewcommand\chapter{\newpage\stdsection}  




% \title{The Use of Landsat-8 for Monitoring of Fresh and Coastal Water}
% \author{
% 	\textsc{Javier A. Concha}%\thanks{Contact Author}
% 	\mbox{}\\
% 	Center for Imaging Science\\ 
% 	Rochester Institute of Technology\\
% 	Rochester, NY, USA\\
% 	\mbox{}\\
% 	\normalsize
% 		\texttt{jxc4005@rit.edu}
% 	}
% \date{}                                           % Activate to display a given date or no date
% %\date{\today}

%%%%%%%%%%% for Appendix
\makeatletter
\newcommand\appendix@section[1]{%
  \refstepcounter{section}%
  \orig@section*{Appendix \@Alph\c@section: #1}%
  \addcontentsline{toc}{section}{Appendix \@Alph\c@section: #1}%
}
\let\orig@section\section
\g@addto@macro\appendix{\let\section\appendix@section}
\makeatother


%%%%%%%%%%%%%%

\usepackage{hyperref}
\hypersetup{
    bookmarks=true,         % show bookmarks bar?
    unicode=false,          % non-Latin characters in Acrobat�s bookmarks
    pdftoolbar=true,        % show Acrobat�s toolbar?
    pdfmenubar=true,        % show Acrobat�s menu?
    pdffitwindow=false,     % window fit to page when opened
    pdfstartview={FitH},    % fits the width of the page to the window
    pdftitle={WV-2 In Water Component Retrieval },    % title
    pdfauthor={Javier Concha},     % author
    pdfsubject={Subject},   % subject of the document
    pdfcreator={Creator},   % creator of the document
    pdfproducer={Producer}, % producer of the document
    pdfkeywords={keyword1} {key2} {key3}, % list of keywords
    pdfnewwindow=true,      % links in new window
    colorlinks=true,       % false: boxed links; true: colored links
    linkcolor=blue,          % color of internal links
    citecolor=green,        % color of links to bibliography
    filecolor=magenta,      % color of file links
    urlcolor=cyan           % color of external links
}

\usepackage[all]{hypcap} % to see figure with hyper ref

\setcounter{secnumdepth}{5}
\setcounter{tocdepth}{5}

%******************************************************************************************************

\begin{document}

\degreetitle{The Use of Landsat-8 for Monitoring of Fresh and Coastal Water}
\degreeauthor{Javier A. Concha}
\degreedate{30 Noviembre 2013}
\prevdegreeA{M.S. Rochester Institute of Technology, 2012}
\advisor{Dr. John R. Schott}
\memberA{Dr. Anthony Vodacek}
\memberB{Dr. Aaron Gerace}
\memberC{Dr. Charles M. Bachmann}
\memberD{Dr. Christy Tyler}

\makeproposaldeclaration
\makePHDproposalapproval
% \makecopyright


% \maketitle

\pagenumbering{roman}

% \chapter*{Abstract}
% \addcontentsline{toc}{chapter}{Abstract}

\begin{abstract}

The Landsat Data Continuity Mission (LDCM; a.k.a. Landsat-8), recently launched (February 2013), is the next generation of Landsat satellite and continues with more than 40 years of continuing imaging acquisition, playing a critical role in monitoring, understanding and managing natural resources such as water. Landsat-8, with its improved spectral bands and radiometric resolution, has the potential to dramatically improve our ability to simultaneously retrieve the three primary coloring agents, chlorophyll (Chl), colored dissolved organic material (CDOM) and suspended material (SM) from water bodies. This work presents the results obtained so far. To date, the method has been tested on Landsat-5 images because Landsat-8 images are still not available.

In the Case 2 water problem, the sensor-reaching signal due to water is very small when compared to the signal due to the atmospheric effects. Therefore, adequate atmospheric correction becomes an important first step to accurately retrieving water parameters. As a first approach, a model based empirical line method (ELM) atmospheric correction method converts sensor-reaching radiance to water leaving reflectance. This model employs pseudo invariant feature (PIF) pixels extraction from Landsat images along with an in water radiative transfer model (HydroLight) to obtain the field spectra. Further atmospheric compensation technique will be investigated.

A look-up-table (LUT) methodology is implemented to retrieve the water parameters. The LUT is created using HydroLight.

Collections of water samples when the satellite passes over the Rochester area are planned for summer 2013. Concentration and inherent optical properties (IOPs) measurement will help to validate the methods.

\end{abstract}

% \chapter*{Acknowledgements}
%*****************************************************************************************************
\begin{acknowledgements}
To my parents. USGS. 
\end{acknowledgements}
%*****************************************************************************************************

\tableofcontents

\listoffigures
\addcontentsline{toc}{chapter}{List of Figures}

\listoftables
\addcontentsline{toc}{chapter}{List of Tables}

% !TEX root=ProposalJavier.tex 
% the previous is to reference main .bib
%% CHAPTER
\chapter{Introduction}
\label{ch:introduction} 
\pagenumbering{arabic} 
Landsat satellites' main mission is to image the land areas of the earth and therefore there are not open ocean (case 1 water) images available. This is the reason why Landsat satellites have been underestimated by the ocean color community for the study of water bodies. This is mainly for its broad bands and low SNR when compared to ocean color satellites such as SeaWiFS and MODIS. However, L8 could fulfill a niche in the ocean community for coastal and in-land water  bodies studies where a 250 m pixel size satellite is not suitable. This is where L8 has a valuable potential for water quality studies in more optically complex water bodies (case 2 waters). Therefore, the overall objective of this research is to demonstrate that the new generation of Landsat satellites, Landsat-8 satellite, hereafter L8, is capable of accurately retrieving water constituents.

The Landsat project has been monitoring the earth for over four decades, being the longest uninterrupted data set available. L8 is the new satellite that continues with this objective. Carrying two instruments onboard, the Operational Land Imager (OLI) and the Thermal InfraRed Scanner (TIRS), L8 is the first of a new generation of Landsat satellite with state-of-the-art technology. With its 12-bit quantization and improved signal-to-noise ratio (SNR), OLI is a big improvement to the Landsat mission. In addition, OLI includes a new coastal band that increased the spectral resolution of the instrument. These improvements are the main drivers to hypothesize that the L8 satellite will definitely have a better performance in the water quality studies than its predecessors. 

The retrieval of water components is {\color{red} in general performed in reflectance domain}, so the very first step in this work is to perform a {\color{red} decent} atmospheric correction to the radiance image from L8. This is a complex task to perform over water because the signal leaving the water that reaches the sensor is too small when compared to the signal reaching the sensor produced by the atmospheric scattering. Most of the atmospheric correction algorithms applied to color ocean satellites are not suitable for highly turbid coastal water \cite{Patt2003}. In this work, two different approaches for atmospheric correction will be investigated. The first one is an ELM-based technique that uses a combination of a radiative transfer model over water and a Landsat reflectance product to determine the black and dark pixels in the image. The second one is an in-scene/band ratio approach, similar to the ones used for atmospherically correcting ocean color satellites (such as SeaWiFS). The water-leaving reflectance values obtained after atmospheric correction are validated by comparison with water surface reflectance measured in situ. 

After having corrected the image, the next step is to apply a retrieval algorithm that outputs water component retrieval maps of the main water components (Chlorophyll, sediment and CDOM). A spectral matching and look-up table (LUT) approach is utilized. It uses a least square error minimization algorithm to find the best match for a specific reflectance signal in a LUT of spectral water-leaving reflectance curves. The LUT is created using HydroLight 5, a radiative transfer model over water. Each curve in the LUT has a specific set of water component concentrations. This is performed in a pixel-by-pixel basis. The concentration values obtained from the retrieval algorithm are validated by comparison with concentration measured in lab from water bodies present in the L8 image.

In order to have outputs that are representative of the water bodies that are being studied, Inherent Optical Properties (IOPs) of those specific waters have to be input to the HydroLight model. To accomplish this, collections of water samples were conducted at the same time that the L8 satellite passed over the area of study (Rochester, NY). After collection, these water samples were analyzed in lab to obtain IOPs for the main water constituents. Furthermore, AOPs and backscattering measurements were also collected for further comparison and pursue closure between HydroLight AOPs results and in-situ AOPs measurements.

% \pagenumbering{arabic}
\chapter{Objectives}
\label{ch:objectives}

As discuss in Chapter \ref{ch:introduction},...

\section{Problem Statement}
\label{sec:problemstatement}
Hypothesis: Landsat-8 can be used to water components retrieval.

\section{Statement of Objectives}
\label{sec:objectives}

\begin{enumerate}
	\item 
	\item
	\item
	\item
\end{enumerate}

\section{Description of Tasks}
\begin{enumerate}
	{\bf \item h }
	{\bf \item h}
	{\bf \item h}
	{\bf \item h}
\end{enumerate}


Atmospheric Compensation for case 2 water. A couple of ways: ELM based and no black target...

Validation. RMS error and field data.
% \subsection{Primary Requirements}
% \subsection{Future Objectives}

\section{Contribution to Field}
\label{contributiontofield}
% !TEX root=ProposalJavier.tex 
% the previous is to reference main .bib
%% CHAPTER
\chapter{Background and Theory}
\label{ch:background}
% -----------------------------------------------------------------------------
\section{IOPs}

Absorption Coefficient.
Scattering Coefficient.

bb, Bb

VSF

Volumen Scattering Phase Function, or $\beta{hat}$ 
% -----------------------------------------------------------------------------
\section{AOPs}
Remote Sensing Reflectance
% -----------------------------------------------------------------------------
\section{Landsat-8}
Landsat-5, Landsat-7
MODIS, VIRSS
% -----------------------------------------------------------------------------
\section{Water Components}
Pure Water. Chl (pigments). CDOM (or DOM). SM or minerals.
% -----------------------------------------------------------------------------
\section{Retrieval Algorithms}
Inversion algorithm 
% -----------------------------------------------------------------------------
\section{Atmospheric Correction}
\subsection{Atmospheric Correction for Ocean Color Satellites}
Atmospheric Correction used for SeaWiFS, MODIS, VIIRS...

\subsection{ELM-based Algorithm}
The {\it empirical line method} (ELM) is a method for calibration of image data to reflectance that uses ground truth.The ELM uses a linear regression in each band to relate digital counts or radiance to reflectance \cite{Schott}. The ground truth can be in general  either control panels or ad hoc control surfaces of known reflectance. These ground truth objects need to be approximately Lambertian to minimize any errors that could be introduced by sensor view angles effects. Also, this calibration targets are assumed flat and level, with no neighboring obscuration, and homogeneous as well. The ELM method generally assumes that the atmosphere is constant over the complete scene. If that is not the case, corrections must be made for changes in the atmosphere over the scene. The regression to be solved for each band in the ELM method is given by
\begin{equation}
	\label{eq:ELM} 
	L = m\times r_d + b
\end{equation}
where $L$ is the radiance reaching the sensor value, $m$ is the slope of the regression, $r_d$ is the reflectance of the target, and $b=L_u$ is the intercept, with $L_u$ the upwelled radiance or path radiance. Then, the reflectance of the any Lambertian objects can be calculated by rearranging \autoref{eq:ELM}. In order to solved the below regression, i.e. determine the value of $m$ and $b$, we need to have two target with known radiance $L$ and reflectance $L_u$. After $m$ and $b$ been determined, the reflectance of each pixel at each wavelength can be calculated from its radiance value from the image.

An ELM target needs to have a size at least three times the ground instantaneous field of view of the sensor that will image it at the time of data collection. Taking this in consideration, the target should be at least 90x90 meters big for the Landsat sensor, which is sometimes difficult to build or even to find in the scene. 

In this research, a modification to the ELM methods is proposed as a calibration tool. This ELM-based method try to avoid the measurement of ground truth at every sensor passover over the scene by using pseudo-invariant features in the scene as one target along with a estimation of water reflectivity for an open lake region for the other target. Pseudo-invariant targets are defined as targets whose reflectivity properties do not change rapidly between different times of collection. Examples of pseudo-invariant target are urban features in the scene. The two targets used in this ELM-based to solve the regression in \autoref{eq:ELM} are referred to in this documents as the {\it bright pixel} and the {\it dark pixel}.

The ELM-based algorithm uses HydroLight for the estimation of the dark pixel field spectra and the Landsat reflectance product (Landsat Surface Reflectance CDR;\cite{LandsatCDR}) for the estimation of the bright pixel spectra data. See \autoref{sec:CDR} for more details about the Landsat reflectance product.

This method employs pseudo-invariant feature (PIF) pixel extraction \cite{Schott:1988} to mask urban landscape from both the reflectance product and the L8 image for the bright pixel determination. The PIF extraction isolates the pseudoinvariant features from the digital imagery. In our case, the PIF are the man-made urban features in a scene. A flowchart of the process is shown in \autoref{fig:PIFflowchart}. The PIF extraction from digital imagery proceeds in the following fashion. An infrared-to-red ratio image is very effective in the classification of water, vegetation, and urban features. The vegetation in this ratio image will tend to have a high brightness when compared to the urban features and water brightness. This infrared-to-red ratio image can be derived from the quotient of the NIR band (band 4 for Landsat-5; band 5 for Landsat-8) and the red band (band 3 for Landsat-5,band 4 for Landsat-8). This ratio image is thresholded from the high digital count values downward to create a mask so the high brightness pixels are eliminated (vegetation pixels) from the image, that is, these pixels are set to a value of zero and the rest (water and urban pixels) to a value of one. The SWIR 2 band (band 7 in Landsat-5 and Landsat-8) is used to eliminate the water pixels from the previous mask since water has nearly zero reflectance in this spectral region. This SWIR 2 band is thresholded from the low brightness values upward. Water pixels will exhibit a low value when compare to the rest of the pixels. A mask is created by assigning a value of zero to the low brightness pixels (water pixels) and a value of one to the rest (urban features and vegetation). Finally, the two mask created are combined using a logical .AND. resulting in a mask that will have a value of one only in the urban feature pixels, i.e. the PIF's. This mask will be named PIF mask for the rest of this document.

The PIF mask is used to determine the bright pixel spectra in both radiance and reflectance values. This is made by applying the PIF mask in ENVI to both the L8 image (in radiance values) and Landsat reflectance product (in reflectance values), and then calculating the statistics. The mean values of the statistics are used as the bright pixel spectra.
% \begin{figure}[H]
% 	\centering
%   \includegraphics[width=120mm]{/Users/javier/Desktop/Javier/PHD_RIT/Latex/Proposal/Images/PIFflowchart.png}
% 	\caption{Illustration of the logic used to segment PIF features. \label{fig:PIFflowchart}}
% \end{figure}
\begin{figure}
	\centering
  \begin{tikzpicture}[node distance=0.75cm, auto]
          \tikzset{
                  basenode/.style={rectangle,rounded corners,draw=black,very thick, inner sep=1em, minimum size=3em, text centered,text width=2cm},
                  productnode/.style={ellipse,rounded corners,draw=black, very thick, text centered,text width=1.5cm},
                  myarrow/.style={->,>=stealth',thick, double = black},
                  mylabel/.style={text width=7em, text centered}
          }
          % SWIR branch
          \node[basenode] (SWIR) {SWIR 2\\ Band};
          \node[basenode, below=of SWIR] (TS1) {Mask by Threshold};
          \node[align=left, right=0.0 of TS1] (C1) {Urban\\Veget.\\Water};
          \node[align=left, right=-0.15 of C1] (C2) {ON\\ON\\OFF};

          % Ratio branch
          \node[basenode, right=2.5cm of SWIR] (Ratio) {Ratio\\ NIR Band/ Red Band};
          \node[basenode, below=of Ratio] (TS2) {Mask by Threshold};
          \node[align=left, right=0.0 of TS2] (C3) {Urban\\Veget.\\Water};
          \node[align=left, right=-0.15 of C3] (C4) {ON\\OFF\\ON};

          % AND
          \path (TS1.south)--(TS2.south) node[pos=.5,below=2cm] (AND) {AND};


          % PIF Mask
          \node[basenode, below=of AND] (PIFMask){PIF Mask};
          \node[align=left, left=0.85 of PIFMask] (C5) {Urban\\Veget.\\Water};
          \node[align=left, right=-0.15 of C5] (C6) {ON\\OFF\\OFF};

          \node[basenode, below=of TS2,right=2.0cm of AND] (Image) {Image};
          \path (Image.south)--(PIFMask.east) node[below=of Image,right=2cm of PIFMask] (AND2) {AND};
          \node[basenode, right=2cm of AND2] (PIFIm){PIF Image};

          \draw[myarrow] (SWIR)--(TS1);
          \draw[myarrow] (Ratio)--(TS2);
          \draw[myarrow] (TS1)--(AND);
          \draw[myarrow] (TS2)--(AND);
          \draw[myarrow] (AND)--(PIFMask);
          \draw[myarrow] (Image)--(AND2);
          \draw[myarrow] (PIFMask)--(AND2);
          \draw[myarrow] (AND2)--(PIFIm);
    
  \end{tikzpicture}
\caption{Illustration of the logic used to segment PIF features. \label{fig:PIFflowchart}}
\end{figure}
The dark pixel spectra in reflectance values is obtained from a HydroLight run with low concentration of water constituents typical of a open lake region. The HydroLight run is spectrally sampled to the L8 sensor response. The dark pixel spectral in radiance values is obtained from a dark region in the lake pixels of the L8 image. This dark region is determined using a statistical analysis to the lake pixels\todo{describe the statistical analysis}. Statistics are computed in this dark region, and the mean values in each band is used as the dark pixel spectra in radiance values.

At the end of this PIF extraction, there are four different spectra: a reflectance (labeled field spectra in ENVI) bright and dark pixel, and a radiance (labeled data spectra in ENVI) bright and dark pixel. These spectra are used in the "Empirical Line" algorithm of the "Calibration Utilities" in ENVI classic to atmospherically correct the L8 image. The product of this process is a image in reflectance values.




\subsection{Band Ratio}


\subsection{TAFKAA}
% -----------------------------------------------------------------------------
\section{Landsat Surface Reflectance CDR}
\label{sec:CDR} 
The Landsat climate data record (CDR) surface reflectance product is part of the higher-level Landsat data product to support land surface change study developed by USGS \cite{LandsatCDR}. The surface reflectance CDR is generated from specialized software called Landsat Ecosystem Disturbance Adaptive Processing System (LEDAPS \todo{Explain LEDAPS} ) \cite{Masek:2006}. The LEDAPS software uses MODIS atmospheric correction routines to correct Level-1 Landsat Thematic Mapper (TM) or Enhanced Thematic Mapper Plus (ETM+) data. Atmospheric variables such as water vapor, ozone, aerosol optical thickness along with geometric variables ({\todo{What is geopotential height?} geopotential height} and digital elevation) are input with Landsat data to the Second Simulation of a Satellite Signal in the Solar Spectrum (6S) radiative transfer model. The 6S model outputs surface reflectance among others parameters. This surface reflectance product is called the Landsat surface reflectance CDR. This Landsat surface reflectance product has comparable uncertainty to the standard MODIS reflectance product \cite{Masek:2006}.

The LEDAPS algorithm works in the following fashion. First, calibrated images from the Landsat satellite are corrected to {\todo{What does TOA reflectance mean?} top-of-atmosphere (TOA) reflectance} by correcting for solar zenith, Sun-Earth distance, TM or ETM+ bandpass, and solar irradiance. Then, the TOA reflectance is atmospherically corrected with the assumptions that the target is Lambertian and infinite, and the gaseous absorption and particle scattering in the atmosphere can be decoupled. The authors expressed the TOA reflectance as
\begin{equation}
	\rho_{TOA}=T_g(O_3,O_2,CO_2,NO_2,CH_4)\\	
		\times \left[\rho_{R+A}+T_{R+A}T_g(H_2O)\frac{\rho_s}{1-S_{R+A}\rho_s}\right]
		\label{eq:TOAref} 
\end{equation}
where $\rho_s$ is the surface reflectance, $T_g$ is the gaseous transmission due to the atmospheric gases, $T_{R+A}$ is Rayleigh and aerosol transmission, $\rho_{R+A}$ is the Rayleigh and aerosols atmospheric intrinsic reflectance, and $S_{R+A}$ is the Rayleigh and aerosols spherical albedo. The 6S radiative transfer code is utilized to compute the transmission, intrinsic reflectance, and spherical albedo terms. Ozone concentrations and column water vapor are derived from ancillary data. The aerosol optical thickness (AOP) is extracted directly from the imagery by using the dark, dense vegetation (DDV) method of Kaufman {\it et al.} \todo{add reference}.This method postulates a linear relation between SWIR surface reflectance and reflectance in the visible bands, based on the physical correlation between chlorophyll absorption and bound water absorption \todo{Research more about AOT and the Kaufman method!!!}. Finally, the derived AOT, ozone, atmospheric pressure, and water vapor are supplied to the 6S radiative transfer algorithm, which then inverts TOA reflectance to surface reflectance using \autoref{eq:TOAref}. 

According to the author in \cite{LandsatCDR}, the Landsat reflectance product has to be used with caution in coastal regions where land area is small relative to adjacent water because the efficacy of the surface correction is likely to be reduced. This product was available only for Landsat 4 TM, Landsat 5 TM and Landsat 7 ETM+ At the time of this publication. An quality assurance (QA) layers is attached to this product and it can be used for pixel-level conditions and validity production. The surface reflectance products is available in the earthexplorer.com website in a HDF-EOS package that contains all necessary files and it can be read in ENVI (through an ENVI header file).
% -----------------------------------------------------------------------------
\section{MODTRAN}
% -----------------------------------------------------------------------------
\section{HydroLight}
HydroLight is a radiative transfer numerical model written in Fortran \cite{MobleHE}. It computes radiance distributions and derived quantities (e.g. irradiances, reflectances, K functions, etc.) for natural water bodies. It was developed by Dr. Curtis Mobley for over 20 years (since 1989) and is a commercial software product of Squoia Scientific, Inc.

% \begin{figure}[H]
% \begin{columns}[onlytextwidth] % contents are top vertically aligned
% 	\column{.35\textwidth}
%   		\includegraphics[height=3cm]{/Users/javier/Desktop/Javier/PHD_RIT/20123_Spring/Modeling/HydroLight/Beamer/absvsz.png}
%   	\column{.32\textwidth}
%   	\footnotesize
%   		\begin{equation}
%   			\cos\theta\frac{dL(z,\theta,\varphi,\lambda)}{dz}=\cdots \notag
%   		\end{equation}
% 	\column{.35\textwidth} 
% 		\includegraphics[height=3cm]{/Users/javier/Desktop/Javier/PHD_RIT/20123_Spring/Modeling/HydroLight/Beamer/RadSpec.png}
% \end{columns}
% \end{figure}
\todo{Try to fix fig.} 
\begin{figure}
% \resizebox{1.0\textwidth}{!}{%
	\centering
  \begin{tikzpicture}[node distance=0.75cm, auto]
          \tikzset{
                  basenode/.style={rectangle,rounded corners,draw=black,very thick, inner sep=1em, minimum size=3em, text centered,text width=2cm},
                  productnode/.style={ellipse,rounded corners,draw=black, very thick, text centered,text width=1.5cm},
                  myarrow/.style={->,>=stealth',thick, double = black},
                  mylabel/.style={text width=7em, text centered}
          }
          %\path[use as bounding box] (0,6) rectangle (4,2);
          \node[basenode] (IOPs) {Inherent Optical Properties};
          \node[basenode, below=of IOPs] (BC) {Boundary Conditions};
          \node[basenode, right=of IOPs] (RTE) {Radiative Transfer Equation};
          \node[basenode, right=of RTE] (rad) {Radiance Distribution};

          \draw[myarrow] (IOPs)--(RTE);
          \draw[myarrow] (BC)-|(RTE);
          \draw[myarrow] (RTE)--(rad);
  \end{tikzpicture}
% } %xobeziser
\caption{Hydrolight flow chart \label{fig:HLflowchart} } 
\end{figure}
\todo{talk about this figure}

The HydroLight physical model has the following characteristics:

\begin{itemize}
	\item It is time-independent.
	\item Horizontally homogeneous IOPs and boundary conditions $\Rightarrow$ one spatial dimension (depth): no restriction on depth dependence of IOPs.
	\item Wavelength between 300 and 1000 nm.
	\item Finite or infinitely deep (non-Lambertian) water-column bottom.
	\item Arbitrary sky radiance onto sea surface.
	\item Cox-Munk air-water surface (parameterizes gravity and capillary waves via the wind speed)
	\item Various bottom boundary options.
	\item Includes all orders of multiple scattering.
	\item It can optionally include Raman scatter by water.
	\item It can optionally include fluorescence by Chl and CDOM.
	\item It can optionally include horizontally homogeneous internal sources such as bioluminescing layers.
	\item Polarization not included.
\end{itemize}

\begin{figure}[H]
	\centering
	\includegraphics[height=6cm]{/Users/javier/Desktop/Javier/PHD_RIT/20123_Spring/Modeling/HydroLight/Beamer/RadianceDef.png}
\caption{Radiance (Figure from \cite{Mobley:2001}) \label{fig:radiance} } 
\end{figure}
\todo{can I use a fig. from other author?} 
where:\\
			\noindent $\Delta Q$: radian energy incident \\
			$\Delta t$: time interval \\
			$\Delta A$: surface area at location (x,y,z)\\
			$\Delta\Omega$: solid angle in direction ($\theta$,$\varphi$) \\
			$\Delta\lambda$: photons wavelength interval
% -----------------------------------------------------------------------------

\begin{equation}
	L(x,y,z,t,\theta,\varphi,\lambda)\equiv\frac{\Delta Q}{\Delta t\Delta A\Delta\Omega\Delta\lambda}~~\left[ Js^{-1}m^{-2}sr^{-1}nm^{-1} \right]
\end{equation}

\begin{equation}
	L(x,y,z,t,\theta,\varphi,\lambda)\equiv\frac{\partial^4 Q}{\partial t\partial A\partial\Omega\partial\lambda}~~\left[ Js^{-1}m^{-2}sr^{-1}nm^{-1} \right]
\end{equation}


{\bf Radiometric Quantities}
\textbf{Spectral downwelling scalar irradiance} at depth z:
\begin{equation}
	E_{od}(z,\lambda)=\int_{2\pi_d} L(z,\theta,\varphi,\lambda)d\Omega~~\left[Wm^{-2}nm^{-1} \right]
\end{equation}
\textbf{Spectral upwelling scalar irradiance} at depth z:
\begin{equation}
	E_{ou}(z,\lambda)=\int_{2\pi_u} L(z,\theta,\varphi,\lambda)d\Omega~~\left[Wm^{-2}nm^{-1} \right]
\end{equation}
\textbf{Spectral scalar irradiance} at depth z:
\begin{align}
	E_{o}(z,\lambda) &\equiv E_{od}(z,\lambda)+E_{ou}(z,\lambda)\\
					 &=\int_{4\pi} L(z,\theta,\varphi,\lambda)d\Omega
\end{align}

\textbf{Spectral downwelling plane irradiance} at depth z:
\begin{equation}
	E_{d}(z,\lambda)=\int_{2\pi_d} L(z,\theta,\varphi,\lambda)|cos\theta|d\Omega~~\left[Wm^{-2}nm^{-1} \right]
\end{equation}
Photosynthetic available radiation, \textbf{PAR}:
\begin{equation}
	PAR(z)\equiv \int_{350nm}^{700nm} \frac{\lambda}{hc}E_o(z,\lambda)d\lambda~~~\left[photons~s^{-1}m^{-2} \right]
\end{equation}

		\begin{figure}[H]
		\centering
		\includegraphics[height=3.5cm]{/Users/javier/Desktop/Javier/PHD_RIT/20123_Spring/Modeling/HydroLight/Beamer/IOPgeo.png}
		\caption{caption \label{label} } 
		\end{figure}

		\textbf{Absorption coefficient:}
		\begin{equation}
			a(\lambda)\equiv \lim_{\Delta r\to 0} \frac{1}{\Phi_i(\lambda)}\frac{\Phi_a(\lambda)}{\Delta r}~~\left[m^{-1} \right]
		\end{equation}
		\textbf{Scattering coefficient:}
		\begin{equation}
			b(\lambda)\equiv \lim_{\Delta r\to 0} \frac{1}{\Phi_i(\lambda)}\frac{\Phi_s(\lambda)}{\Delta r}~~\left[m^{-1} \right]
		\end{equation}
		\textbf{Attenuation coefficient:}
		\begin{equation}
			c(\lambda)=a(\lambda)+b(\lambda)~~\left[m^{-1} \right]
		\end{equation}

{\bf Inherent Optical Properties}
Contributions by the various components to the absorption coefficient
\begin{figure}[H]
\centering
  		\includegraphics[height=5cm]{/Users/javier/Desktop/Javier/PHD_RIT/20123_Spring/Modeling/HydroLight/Beamer/AbsCoeff.png}
\end{figure}


\begin{figure}[H]
\centering
  		\includegraphics[height=5cm]{/Users/javier/Desktop/Javier/PHD_RIT/20123_Spring/Modeling/HydroLight/Beamer/ScatCoeff.png}
\end{figure}


Volume scattering function, \textbf{VSF}:
\begin{equation}
	\beta(\psi,\lambda)\equiv \lim_{\Delta r\to 0} \lim_{\Delta \Omega\to 0}  \frac{\Phi_s(\psi,\lambda)}{\Phi_i(\lambda)\Delta r\Delta \Omega}~~\left[m^{-1}sr^{-1} \right]
\end{equation}
but $\Phi_s(\psi,\lambda)=I_s(\psi,\lambda)\Delta \Omega$ and $E_i(\lambda)=\Phi_i(\lambda)/\Delta A \Rightarrow$
\begin{equation}
	\beta(\psi,\lambda)= \lim_{\Delta V\to 0} \frac{I_s(\psi,\lambda)}{E_i(\lambda)\Delta V}~~\left[m^{-1}sr^{-1} \right]
\end{equation}
with $\Delta V=\Delta r\Delta A$

Scattering coefficient from VSF:
\begin{equation}
	b(\lambda)=\int_{4\pi} \beta(\psi,\lambda)d\Omega=2\pi\int_0^\pi \beta(\psi,\lambda)sin\psi d\psi
\end{equation}
And \textbf{backscatter coefficient}:
\begin{equation}
	b_b(\lambda)\equiv 2\pi\int_{\pi/2}^\pi \beta(\psi,\lambda)sin\psi d\psi
\end{equation}


\begin{figure}[H]
\centering
  		\includegraphics[height=4cm]{/Users/javier/Desktop/Javier/PHD_RIT/20123_Spring/Modeling/HydroLight/Beamer/VSFweb.png}
\end{figure}
Example VSFs:\\
\textcolor{blue}{blue curve} : clear, open ocean water ($\lambda = 514 nm$)\\
\textcolor{green}{green curve} : harbor ($\lambda = 514 nm$)\\
\textcolor{red}{red curve} : very productive coastal waters ($\lambda = 530 nm$)

\textbf{Backscatter fraction:}
\begin{equation}
	B_b=\frac{b_b}{b}
\end{equation}
\textbf{Single-scattering albedo:}
\begin{equation}
	\omega_o=\frac{b(\lambda)}{c(\lambda)}
\end{equation}
\textbf{Volumen scattering phase function:}
\begin{equation}
	\tilde{\beta}(\psi,\lambda)\equiv \frac{\beta(\psi,\lambda)}{b(\lambda)}~~\left[sr^{-1} \right]\Rightarrow \beta(\psi,\lambda)=b(\lambda)\tilde{\beta}(\psi,\lambda)
\end{equation}

{Apparent Optical Properties}
\textbf{AOPs:} depend both on the medium and on the directional structure of the ambient light field

\vspace{\baselineskip}

\textbf{Irradiance reflectance:}
\begin{equation}
	R(z,\lambda)\equiv \frac{E_u(z,\lambda)}{E_d(z,\lambda)}
\end{equation}
\textbf{Remote sensing reflectance:}
\begin{equation}
	R_{rs}(\theta,\varphi,\lambda)\equiv \frac{L_w(\theta,\varphi,\lambda)}{E_d(\lambda)}~~\left[sr^{-1} \right]
\end{equation}
where $L_w$ is the \textbf{water-leaving radiance}

Under typical oceanic conditions, irradiances and radiances decrease exponentially with depth, then
\begin{equation}
	E_d(z,\lambda)\equiv E_d(0,\lambda) exp\left[-\int_0^{z}K_d(z',\lambda)dz'\right]
\end{equation}
where $K_d(z,\lambda)$ is the \textbf{spectral diffuse attenuation coefficient} for spectral downwelling plane irradiance. 
\begin{align}
	K_d(z,\lambda) 	&=- \frac{d\ln E_d(z,\lambda)}{dz} \notag \\
					&=-\frac{1}{E_d(z,\lambda)}\frac{dE_d(z,\lambda)}{dz} ~~\left[m^{-1} \right]
\end{align}

\textbf{Note:} IOPs are additive whereas AOPs are not

{Optical Constituents of Water}
\begin{itemize}
	\item Water
	\item Dissolved Organic Compounds (CDOM)
	\item Organic Particles (Chlorophyll)
	\item Inorganic Particles (Suspended Matter)
\end{itemize}

\textbf{Total IOP:}

\begin{equation}
	a_{total}(z,\lambda) = \sum_{i=1}^{ncomp} a_i(z,\lambda)
\end{equation}
e.g.,
\begin{equation}
	a_{total}(z,\lambda) =  a_w(\lambda) +a_{CDOM}(z,\lambda)+ a_{Chl}(z,\lambda)+a_{SM}(z,\lambda) \notag
\end{equation}

{The Radiative Transfer Equation}
\textbf{RTE} expresses conservation of energy for a collimated beam of radiance traveling through an absorbing, scattering and emitting medium

\vspace{\baselineskip}

The \textbf{RTE}:
\begin{align}
	\cos\theta\frac{dL(z,\theta,\varphi,\lambda)}{dz}&=-c(z,\lambda)L(z,\theta,\varphi,\lambda)\cdots \notag \\
	&+\int_{4\pi} L(z,\theta',\varphi',\lambda)\beta(z;\theta',\varphi' \to \theta,\varphi;\lambda)d\Omega'\cdots \notag  \\
	&+S(z,\theta,\varphi,\lambda)~~\left[W~m^{-3}sr^{-1}nm^{-1} \right]
\end{align}

\begin{figure}[H]
\centering
  		\includegraphics[height=4cm]{/Users/javier/Desktop/Javier/PHD_RIT/20123_Spring/Modeling/HydroLight/Beamer/RTE1.png}
\end{figure}

\begin{equation}
	-c(z,\lambda)L(z,\theta,\varphi,\lambda)\cdots \notag 
\end{equation}

\begin{figure}[H]
\centering
  		\includegraphics[height=4cm]{/Users/javier/Desktop/Javier/PHD_RIT/20123_Spring/Modeling/HydroLight/Beamer/RTE2.png}
\end{figure}

\begin{equation}
+\int_{4\pi} L(z,\theta',\varphi',\lambda)\beta(z;\theta',\varphi' \to \theta,\varphi;\lambda)d\Omega'\cdots \notag
\end{equation}

{\centering
\begin{equation}
	S(z,\theta,\varphi,\lambda) \notag
\end{equation}}

\begin{itemize}

\item {$S$ account for bioluminescence or inelastic-scattering processes}

\vspace{\baselineskip}

\item For example, the model is first run at the wavelength of excitation, $\lambda_{ex}$ to compute the energy shifted by inelastic scattering from $\lambda_{ex}$ to another wavelength $\lambda$, and the model is run again at $\lambda$, with the radiance shifted from $\lambda_{ex}$ appearing as a source term $S$ at $\lambda$

\end{itemize}

The \textbf{RTE}:
\begin{align}
	\cos\theta\frac{dL(z,\theta,\varphi,\lambda)}{dz}&=-c(z,\lambda)L(z,\theta,\varphi,\lambda)\cdots \notag \\
	&+\int_{4\pi} L(z,\theta',\varphi',\lambda)\beta(z;\theta',\varphi' \to \theta,\varphi;\lambda)d\Omega'\cdots \notag  \\
	&+S(z,\theta,\varphi,\lambda)~~\left[W~m^{-3}sr^{-1}nm^{-1} \right]
\end{align}
% ----------------------------------------------------------------------------------
\section{Water Quality Parameters Retrieval}
\cite{Jensen}
\cite{Mustard2001}
\subsection{Governing Equation}

\hl{The comparison is performed in the reflectance domain.}

\hl{CHL, SM and CDOM explanation}

From the total energy coming from the sun, only approximately 1390 $\left[\frac{W}{m^2}\right]$ reaches the Earth's atmosphere \cite{Schott}. This integrated value is known as the \emph{exoatmospheric irradiance}, or $E_S'$, and represents the total energy per unit area just outside the Earth's atmosphere due to the solar energy. Recall that \emph{irradiance} is the rate at which the radiant flux ($\Phi$) is delivered to a surface ($A$), defined as

\begin{equation} \label{eq:irradiance}
E = \frac{d\Phi}{dA}   \indent   \indent  \left[\frac{W}{m^2}\right]  
\end{equation} 

So, the $E_S'$ is calculated assuming that the flux $\Phi$ comes from a point source at the center of the sun such that it would produce an existence at the sun's surface, producing a flux at the mean Earth-sun distance of 1390 $\left[\frac{W}{m^2}\right]$ . For the present work, it is more convenient to express the irradiance spectrally, or other words as a function of wavelength, so we can describe the energy at desired wavelength, or spectral band.

The light source, usually the sun, interact with the target and then reach the sensor. This interaction will help us to extract about the target, in this case the water body. That is why in order to understand how the water quality parameter are retrieved, first it is necessary to introduce the concept of sensor reaching radiance. The sensor reaching radiance is defined as the accumulation of photons at the front of a sensor that one wishes to collect in an effort to obtain information about the target \cite{GeraceThesis} . The total sensor-reaching radiance is the sum of the radiances due to the individual solar and thermal paths.  \cite{Schott} shows that in the VNIR region (approximately 0.3-2.5 [$\mu m$]), the solar energy is so many orders of the magnitude higher than the self-emitted energy, so the thermal paths are negligible for this study. Also, we consider that radiance from the background is negligible because the water bodies are typically several kilometers wide. Assuming that the target is approximately Lambertian (radiance is equal in all directions), the total sensor-reaching radiance, $L$, is defined as

\begin{equation} \label{eq:gov1}s
L(\lambda) = \frac{E'_S(\lambda)cos(\sigma')r(\lambda)\tau_1(\lambda)\tau_2(\lambda)}{\pi} +
                        \frac{E_{ds}(\lambda)r(\lambda)\tau_2(\lambda)}{\pi} + L_{us}(\lambda)
\end{equation} 
where:
\begin{tabbing}
\indent \indent \indent  $L(\lambda)$ \hspace{1mm}\=:  \indent \= total sensor-reaching radiance\\
\indent \indent \indent  $E'_S(\lambda)$\>: \>exoatmospheric spectral irradiance\\
\indent \indent \indent $\sigma'$\>:\>solar-zenith angle\\
\indent \indent \indent $r(\lambda)$\>:\>spectral reflectance target\\
\indent \indent \indent $\tau_1(\lambda)$\>:\>Sun-target path transmission of atmosphere\\
\indent \indent \indent $\tau_2(\lambda)$\>:\>target-sensor path transmission of atmosphere\\
\indent \indent \indent $E_{ds}(\lambda)$\>:\>solar downwelling irradiance\\
\indent \indent \indent $L_{u}(\lambda)$\>:\>solar upwelling irradiance\\
\end{tabbing}

Equation \eqref{eq:gov1} is the solar term of the "big equation" described by \cite{Schott}.

\subsection{Constituent Retrieval}

When imaging water body, a set of new paths need to be taken in account to determine the sensor-reaching radiance.
 \begin{figure}[H]
	\centering
    	\includegraphics[width=100mm]{/Users/javier/Desktop/Javier/MASTER_RIT/2011_THESIS/LaTeX/Thesis/Images/WaterColumn.eps}
 	\caption{Contributions to sensor-reaching radiance from the water column \label{fig:WaterColumn}}
  \end{figure}


Each sensor-reaching radiance curve is associated with a specific combination of water components (CHL, SM and CDOM). HidroLight provides us the \hl{remote sensing reflectance}, $R_{RS}$, which describes how much of the total incident downwelling irradiance is ultimately returned from a water column in a given direction, defined as

\begin{equation} \label{eq:Rrs}
R_{RS}(\theta,\phi,\lambda,z=a) = \frac{L(\theta,\phi,\lambda,z=a)}{E_d(\lambda,z=a)}   \indent   \indent  \left[\frac{1}{sr}\right]  
\end{equation} 
where:
\begin{tabbing}
\indent \indent \indent  $\theta$ \hspace{1.5mm}\=:  \indent \= sensor-zenith angle\\
\indent \indent \indent  $\phi$\>: \>sensor-azimuth angle\\
\indent \indent \indent $L$\>:\>water-leaving radiance\\
\indent \indent \indent $E_d$\>:\>total downwelling irradiance\\
\indent \indent \indent $\lambda$\>:\>wavelength dependent\\
\indent \indent \indent $a$\>:\>height just above the water's surface\\
\end{tabbing}



\section{Shallow Water Retrieval}
\section{LUT Method}
\subsection{Population}
\subsection{Optimization Algorithm  - lsqnonlin}

\section{State of the Research}

As shown in \cite{GeraceThesis}
and \cite{Mobley} and \cite{Lesser}
%% CHAPTER
\chapter{Methodology and Approach}
\section{OLI Sensor}
\subsection{Sensor Response}
Noise level. Quantization. SNR.

\section{Retrieval Process}
\subsection{The Look-Up Table}


\subsubsection{Real Atmosphere Conditions}



\section{Over-Water Atmospheric Compensation}
\subsection{The Empirical Line Method (ELM)}
\subsection{LUT Method}
% !TEX root=ProposalJavier.tex 
% the previous is to reference main .bib
%% CHAPTER
\chapter{Initial Results}
\section{Synthetic Data}
\subsection{Deep Water}

  \begin{figure}[H]
	\centering
    	\includegraphics[width=100mm]{/Users/javier/Desktop/Javier/MASTER_RIT/2011_THESIS/LUT/LUT_1/Images/LUT1_120re.eps}
 	\caption{No Error  \label{fig:errorLUT1}}
  \end{figure}

As you can see in Figure \ref{fig:errorLUT1}...

  \begin{figure}[H]
	\centering
    	\includegraphics[width=100mm]{/Users/javier/Desktop/Javier/MASTER_RIT/2011_THESIS/LUT/LUT_1/Images/errorCHLSMCDOM.eps}
 	\caption{Error  \label{fig:errorLUT1}}
  \end{figure}


\subsection{Sensitivity Analysis}

\section{Real Data}
\input{Future_Work.tex}
% !TEX root=ProposalJavier.tex 
% the previous is to reference main .bib
%% CHAPTER
\appendix
\chapter*{Appendix}
\addcontentsline{toc}{chapter}{Appendix}
\section{Landsat-8 Sensor Characteristics}
\section{HydroLight Specification}
\section{MODTRAN Specification}

  
%\bibliographystyle{ieeetr}
%\bibliographystyle{unsrtnat}
\bibliographystyle{apalike}

\bibliography{/Users/javier/Desktop/Javier/PHD_RIT/Latex/javier_bib}


\end{document}  