% !TEX root=ProposalJavier.tex 
% the previous is to reference main .bib
%% CHAPTER
\chapter{Initial Results}
\section{Synthetic Data}

Preliminary results of the first atmospheric correction methods for an area of water in the Rochester, NY region are shown in Figure~\ref{fig:waterpxs}. This figure shows the spectrum of the water pixels after atmospheric correction (in reflectance values), and these exhibit shapes that correspond with the shapes of typical water pixels. Figure~\ref{fig:refcomp} shows water surface reflectance spectra as preliminary results from the model-based ELM method (solid lines) compared with results from a traditional ELM method (dashed lines) for four different regions of interest (ROIs) in the Rochester area (Cranberry Pond, Long Pond, and nearshore and offshore of the Lake Ontario). As can be seen, the atmospheric correction algorithm proposed exhibits less than one reflectance unit of difference in comparison to the results from the traditional ELM algorithm. While small, these differences have a significant impact on the retrieved water constituents. Therefore, the proposed atmospheric correction algorithm will be improved by this research. A further validation with ground-truth data is anticipated at the end of this research.
\vspace{-.3cm}
\begin{figure}[htb]
  \begin{minipage}[c]{0.48\linewidth}
    \centering
      \includegraphics[height=6cm]{WaterPixels_2.eps}
      \caption{Water pixel spectra after applying the model-based ELM atmospheric correction method.}
      \label{fig:waterpxs}
    % \vspace{1.5cm}
    % \centerline{(a)}\medskip
  \end{minipage}
  \hfill
  \begin{minipage}[d]{0.48\linewidth}
    \centering
      \includegraphics[height=6cm]{WaterPixelComparisonELMELMbased}
      \caption{Comparison between traditional ELM (dashed lines) and model-based ELM (solid lines).}
      \label{fig:refcomp}
    % \vspace{1.5cm}
    % \centerline{(b)}\medskip
  \end{minipage}
  % \caption{Water pixel spectra: (a) after atmospheric correction and (b) comparison with traditional ELM method.}
\end{figure}


\subsection{Deep Water}

  \begin{figure}[H]
	\centering
    	\includegraphics[width=100mm]{/Users/javier/Desktop/Javier/MASTER_RIT/2011_THESIS/LUT/LUT_1/Images/LUT1_120re.eps}
 	\caption{No Error  \label{fig:errorLUT1}}
  \end{figure}

As you can see in Figure \ref{fig:errorLUT1}...

  \begin{figure}[H]
	\centering
    	\includegraphics[width=100mm]{/Users/javier/Desktop/Javier/MASTER_RIT/2011_THESIS/LUT/LUT_1/Images/errorCHLSMCDOM.eps}
 	\caption{Error  \label{fig:errorLUT1}}
  \end{figure}


\subsection{Sensitivity Analysis}

\section{Real Data}