% !TEX root=ProposalJavier.tex 
% the previous is to reference main .bib
%% CHAPTER
\chapter{Initial Results}
% ------------------------------
\section{Atmospheric Correction}

Preliminary results of the model-based ELM atmospheric correction methods for different water bodies in the Rochester Embayment area are shown in \autoref{fig:waterpxs}. This figure shows the spectrum of the water pixels in reflectance values after applying the model-based ELM atmospheric correction. These curves exhibit shapes that correspond with the shapes of typical water pixels. \autoref{fig:refcomp} shows water reflectance spectra as preliminary results from the model-based ELM method (solid lines) compared with results from a traditional ELM method (dashed lines) for four different ROIs in the Rochester Embayment area (Cranberry Pond, Long Pond, and nearshore and offshore of the Lake Ontario, labeled as Cranb, LongS, OntNS and OntOS in \autoref{fig:0910913Sites}, respectively). 

The traditional ELM method was performed with reflectance measurements taken in the field. A reflectance measurement taken of the sand of Charlotte Beach, Rochester, NY (labeled as SandDry in \autoref{fig:0910913Sites}) was used for the bright pixel while a reflectance measurement taken at the site OntNS was used for the dark pixel. The radiance values were taken from the corresponding ROIs in the Landsat 8 image. 

As can be seen in \autoref{fig:refcomp}, the atmospheric correction algorithm proposed in this study exhibits less than one percent reflectance unit ($<0.01\Rightarrow <1\%$) of difference in comparison to the results from the traditional ELM algorithm.

\begin{figure}[!ht]
  \centering
  \includegraphics[width=12cm]{/Users/javier/Desktop/Javier/PHD_RIT/Latex/Proposal/Images/groundtruth-sitenames-no-ends.jpg}
  \caption{Sites in the Rochester Embayment for the water sample collection on September, $19^{th}$, 2013.\label{fig:0910913Sites} } 
  \vspace{1cm}
\end{figure}

% \vspace{-.3cm}
\begin{figure}[!ht]
  \begin{minipage}[c]{0.48\linewidth}
    \centering
      \includegraphics[height=6cm]{/Users/javier/Desktop/Javier/PHD_RIT/ConferencesAndApplications/NESSF14/latex/WaterPixels_2.eps}
      \caption{Water pixel spectra after applying the model-based ELM atmospheric correction method.}
      \label{fig:waterpxs}
    % \vspace{1.5cm}
    % \centerline{(a)}\medskip
  \end{minipage}
  \hfill
  \begin{minipage}[d]{0.48\linewidth}
    \centering
      \includegraphics[height=6cm]{/Users/javier/Desktop/Javier/PHD_RIT/ConferencesAndApplications/NESSF14/latex/WaterPixelComparisonELMELMbased}
      \caption{Comparison between traditional ELM (dashed lines) and model-based ELM (solid lines).}
      \label{fig:refcomp}
    % \vspace{1.5cm}
    % \centerline{(b)}\medskip
  \end{minipage}
  % \caption{Water pixel spectra: (a) after atmospheric correction and (b) comparison with traditional ELM method.}
\end{figure}


% ------------------------------
\section{Retrieval}

Preliminary results for concentration maps for each CPA over the Rochester Embayment, Rochester, NY are shown in Figure~\ref{fig:retrievalresults}. The expected trend of having low concentration of CPAs in the offshore of Lake Ontario and higher concentrations in the nearby ponds (Long Pond and Cranberry Pond) can be seen. 
\begin{figure}[htb]
\centering
\includegraphics[trim=200 100 180 0,clip,height=9cm]{/Users/javier/Desktop/Javier/PHD_RIT/ConferencesAndApplications/2014_RITResearchSymposium/Images/RetrievalResults.eps}
   \caption{Retrieval preliminary results.}
      \label{fig:retrievalresults}   
\end{figure}

Comparisons between retrieved CPAs concentrations and field measurements are shown in Figure~\ref{fig:chlcomp}, Figure~\ref{fig:tsscomp} and Figure~\ref{fig:cdomcomp} for four different stations in the area of study. This comparison with field measurements showed good agreement at low concentrations but differences at higher concentrations. Ongoing work is focusing on incorporation of the IOP differences between water bodies in the LUT optimization process.
\begin{figure}[htb]
\centering
    \includegraphics[height=7cm]{/Users/javier/Desktop/Javier/PHD_RIT/ConferencesAndApplications/IGARSS2014/paper/Images/chlcomp.eps} 
    \caption{Comparison between measured and retrieved chlorophyll {\it a} concentration.}
    \label{fig:chlcomp} 
\end{figure}     

\begin{figure}[htb]
\centering
    \includegraphics[height=7cm]{/Users/javier/Desktop/Javier/PHD_RIT/ConferencesAndApplications/IGARSS2014/paper/Images/tsscomp.eps}   
    \caption{Comparison between measured and retrieved TSS concentration.}
    \label{fig:tsscomp} 
\end{figure}  

\begin{figure}[htb]
\centering
    \includegraphics[height=7cm]{/Users/javier/Desktop/Javier/PHD_RIT/ConferencesAndApplications/IGARSS2014/paper/Images/cdomcomp.eps}    
    \caption{Comparison between measured and retrieved CDOM concentration.}
    \label{fig:cdomcomp} 
\end{figure}  

\section{Concluding Remarks}